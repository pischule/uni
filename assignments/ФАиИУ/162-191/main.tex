% !TeX spellcheck = ru_RU
\documentclass[a4paper]{article}
\usepackage[T2A]{fontenc}
\usepackage[utf8]{inputenc}
\usepackage[russian]{babel}
\usepackage{geometry}
\begin{document}
\begin{titlepage}


  \newgeometry{margin=1cm}
  
  \centerline{\large \bf МИНИСТЕРСТВО ОБРАЗОВАНИЯ РЕСПУБЛИКИ БЕЛАРУСЬ}
  \bigskip
  \bigskip
  \centerline{\large \bf БЕЛОРУССКИЙ ГОСУДАРСТВЕННЫЙ УНИВЕРСИТЕТ}
  \bigskip
  \bigskip
  \centerline{\large \bf ФАКУЛЬТЕТ ПРИКЛАДНОЙ МАТЕМАТИКИ И ИНФОРМАТИКИ}
  \vfill
  \vfill
  \vfill
  \centerline{\Large \bf ФУНКЦИОНАЛЬНЫЙ АНАЛИЗ И ИНТЕГРАЛЬНЫЕ УРАВНЕНИЯ}
  \bigskip
  \bigskip
  \vfill
  \begin{centering}
    {\large
    Домашняя работа\\
    студента 2 курса 2 группы \\}
  \end{centering}
  \centerline{\large \bf Пищулёнка Максима Сергеевича}
  \vfill
  \vfill
  \hfill
  \begin{minipage}{0.25\textwidth}
    {\large{\bf Преподаватель} \\
  {\it Дайняк Виктор \\ Владимирович}}
  \end{minipage}
  \vfill
  \vfill
  \centerline{\Large \bf Минск 2020}
  
  \end{titlepage}

\begin{center}
\end{center}

\leavevmode\thispagestyle{empty}\newpage

\section*{162}
$f(x) = \sqrt{x}$\\
\textbf{Решение}\\
Не выполняется условие 2 при $\lambda < 0$\\
\textbf{Ответ}\\
Нет.
\section*{163}
$f(x)=\sqrt{|x|}$\\
\textbf{Решение}\\
Не выполняется условие 2 при $|\lambda| \neq 1$\\
\textbf{Ответ}\\Нет.

\section*{164}
$f(x) = |x-1|$\\
\textbf{Решение}\\
Не выполняется условие 1: $x = 0, \ \|x\| = |x - 1| = |0 - 1| = 1 \ne 0$\\
\textbf{Ответ}\\Нет.

\section*{165}
$f(x) = \sqrt{x^2}$\\
\textbf{Решение}\\
$\sqrt{x^2} = |x|$. Модуль является нормой\\
\textbf{Ответ}\\Да.

\section*{166}
$f(x) = 5 |x|$\\
\textbf{Решение}\\
Если $\|x\|$ - норма, то для положительных $\alpha$: $\alpha\|x\|=0 \Leftrightarrow \|x\|=0$. $\|\lambda \alpha x\| = |\lambda\|\alpha\||x\| = |\lambda\||\alpha x\|$. 
3. Неравенство можно домножать на коэффициент.\\
\textbf{Ответ}\\Да.

\section*{167}
$f(x) = x^2$\\
\textbf{Решение}\\
$\lambda = 2,\ x = 2$:  $(2*2)^2 \neq 2*2^2$\\
\textbf{Ответ}\\Нет.

\section*{168}
$f(x) = |\arctan x| $\\
\textbf{Решение}\\
$\lambda = 3,\ x = \frac{\sqrt{3}}{3}$:  $|arctg(3*\frac{\sqrt{3}}{3})| = \frac{\pi}{3} \neq 3*|arctg(\frac{\sqrt{3}}{3})| = \frac{\pi}{2}$\\
\textbf{Ответ}\\Нет.

\section*{169}
$f(x) = \ln |x| $\\
\textbf{Решение}\\
$\lambda = 2,\ x = e$:  $\ln 2e \neq 2\ln e$\\
\textbf{Ответ}\\Нет.

\section*{170}
$f(\overline{a}) = \sqrt{xy}$\\
\textbf{Решение}\\
$a = [0,1], b = [1,0]. \|a + b\| = 1,\|a\|=0, \|b\|=0$. То есть не выполняется 3 условие.\\
\textbf{Ответ}\\Нет.

\section*{171}
$f(\overline{a}) = |x| + |y|$\\
\textbf{Решение}\\
\rule{5mm}{0pt}1. Сумма модулей равна 0 $\Leftrightarrow$ все модули равны 0.\\
\rule{5mm}{0pt}2. $|\lambda x|+|\lambda y| = |\lambda|(|x|+|y|).$ \\
\rule{5mm}{0pt}3. Модуль суммы не больше суммы модулей.\\
\textbf{Ответ}\\Да.

\section*{172}
$f(\overline{a}) = \max (|x| + |y|)$\\
\textbf{Решение}\\
Первые два условия очевидны. Рассмотрим третье.\\ $max(|x_1+x_2|,|y_1+y_2|) \le max(|x_1|+|x_2|,|y_1|+|y_2|).$ Без ограничения общности считаем $|x_1| \le |y_1|$. \\Пусть $|x_2| \le |y_2|$. Тогда $max(|x_1|+|x_2|,|y_1|+|y_2|) = |y_1|+|y_2| = max(|x_1|,|y_1|) + max(|x_2|,|y_2|).$ \\Пусть теперь $|y_2| \le |x_2|$. Тогда либо $max(|x_1|+|x_2|,|y_1|+|y_2|) = |x_1|+|x_2| \le |y_1|+|x_2| = max(|x_1|,|y_1|) + max(|x_2|,|y_2|)$, либо $max(|x_1|+|x_2|,|y_1|+|y_2|) = |y_1|+|y_2| \le |y_1|+|x_2| = max(|x_1|,|y_1|) + max(|x_2|,|y_2|)$.\\
\textbf{Ответ}\\Да.

\section*{173}
$f(\overline{a}) = \sqrt{x^2 + y^2} + \sqrt{xy}$\\
\textbf{Решение}\\
$a = [0,1], b = [1,0]. \|a + b\|=\sqrt{2}+1>\|a\|+\|b\|=2$\\
\textbf{Ответ}\\Нет.

\section*{174}
$f( \overline{a}) = |x^2 - y^2| $\\
\textbf{Решение}\\
$\lambda = 2,\ x = e$:  $\ln 2e \neq 2\ln e$\\
\textbf{Ответ}\\Нет.

\section*{175}
$f(\overline{a}) = \sqrt[3]{x^6 + y^6} $\\
\textbf{Решение}\\
$\lambda = 2, x = (1, 1)$: $\|\lambda x\| = 4\sqrt[3]{2} \neq 2\|x\| = 2\sqrt[3]{2}$\\
\textbf{Ответ}\\Нет.

\section*{176}
$\max_{t\in [a, b]} (x(t))$\\
\textbf{Решение}\\
Максимум модуля = 0 $\Leftrightarrow$ подмодульное выражение = 0. Коэффициент легко выносится за максимум. Максимум суммы не больше суммы максимумов.\\
\textbf{Ответ}\\Да.

\section*{177}
$\max_{t\in [a, b]} (x'(t))$\\
\textbf{Решение}\\
$x(t) = 3, \|x\| = 0$\\
\textbf{Ответ}\\Нет.

\section*{178}
$|x(b) - x(a)| + \max_{t\in [a, b]} |x'(t)|$\\
\textbf{Решение}\\
$x(t) = 3, \|x\| = 0$\\
\textbf{Ответ}\\Нет.

\section*{179}
$\int_a^b |x(t)|dt + \max_{t\in [a, b]} |x'(t)|$\\
\textbf{Решение}\\
Интеграл от неотрицательной функции равен 0 $\Leftrightarrow$ функция тождественно равна 0. Так как из интеграла и производной можно выносить константы, выполняется второе условие. А так как производная суммы равна сумме производных, модуль суммы не больше суммы модулей и максимум суммы не больше суммы максимумов, а также учитывая, что интеграл суммы равен сумме интегралов, имеет место неравенство 3\\
\textbf{Ответ}\\Да.

\section*{180}
$|x(a)| + \max_{t\in [a,b]} |x'(t)|$\\
\textbf{Решение}\\
В первом условии при $\|x\| = 0$ из равенства нулю второго слагаемого имеем x(t) = C, из равенства нулю первого получаем C = 0. Второе условие очевидно выполняется, так как константа выносится за модуль, максимум, производную. Третье выполняется, так как модуль суммы не больше суммы модулей, а производная суммы равна сумме производных.\\
\textbf{Ответ}\\Да.

\section*{178}
$|x(b) - x(a)| + \max_{t\in [a, b]} |x'(t)|$\\
\textbf{Решение}\\
$x(t) = 3, \|x\| = 0$\\
\textbf{Ответ}\\Нет.


Для \textbf{182-184} проверим, могут ли расстояния в $C[a,b]$ и $L_2[a,b]$ выступать в качестве норм. Для этого достаточно проверить второе условие (остальные условия выполняются за счет того, что совпадают с условиями задания расстояния).\\ Из максимума и модуля коэффициент лямбда легко выносится, следовательно в $C[a,b]$ расстояние можно рассматривать как норму. \\$\|\lambda x\| = (\int_a^b(\lambda x(t))^2dt)^{\frac{1}{2}} = (\int_a^b\lambda^2 (x(t))^2dt)^{\frac{1}{2}} = |\lambda|(\int_a^b(x(t))^2dt)^{\frac{1}{2}} = |\lambda| \|x\|$. Итого в $L_2[a,b]$ тоже норма.\\
Так же из предыдущих заданий убедились, что модуль также может выступать в качестве нормы на множестве действительных чисел.\\
То есть для этих трех норм выполняются все условия норм. Причем несложно заметить, что все эти нормы неотрицательны при любых аргументах (максимум модуля, интеграл квадрата и собственно сам модуль, все они не бывают меньше 0). Будем этим пользоваться.\\\\

\section*{182}
$\| x'' \|_{C[a, b]} + \| x \|_{\tilde{L_2}[a, b]}$\\
\textbf{Решение}\\
$\|x\|=\|x''\|_{C[a,b]}+\|x\|_{L_2[a,b]}=0 \Leftrightarrow \|x''\|_{C[a,b]}=0, \|x\|_{L_2[a,b]}=0 \Leftrightarrow x(t) = 0$.\\ $\|\lambda x\|=\|(\lambda x)''\|_{C[a,b]}+\|\lambda x\|_{L_2[a,b]} = \|\lambda x''\|_{C[a,b]}+\|\lambda x\|_{L_2[a,b]} =\\=  |\lambda\||x\|$.\\ $\|x+y\|=\|(y+x)''\|_{C[a,b]}+\|y+x\|_{L_2[a,b]} = \|y''+x''\|_{C[a,b]}+|y+x\|_{L_2[a,b]} \le \|x\|+\|y\|$.\\
\textbf{Ответ}\\Да.

\section*{183}
$|x(a)| + |x(b)| + \| x'' \|_{C[a, b]}$\\
\textbf{Решение}\\
$\|x\|=\|x''\|_{C[a,b]}+|x(a)|+|x(b)|=0 \Leftrightarrow \|x''\|_{C[a,b]}=0, |x(a)|=0, |x(b)|=0 \Leftrightarrow x''=0, x(a)=0, x(b)=0\Leftrightarrow x=Cx+C_0, x(a)=0, x(b)=0 \Leftrightarrow  x(t)=0$.\\ $\|\lambda x\|=\|\lambda x''\|_{C[a,b]}+|\lambda x(a)|+|\lambda x(b)| =|\lambda\||x\|$.\\ $\|x+y\|=\|(y+x)''\|_{C[a,b]}+|y(a)+x(a)|+|y(b)+x(b)| = \|y''+x''\|_{C[a,b]}+|y(a)+x(a)|+|y(b)+x(b)| \le \|x\|+\|y\|$.\\
\textbf{Ответ}\\Да.

\section*{184}
$|x(a)| +  + \| x'' \|_{C[a, b]} + \| x'' \|_{\tilde{L_2}[a, b]}$\\
\textbf{Решение}\\
$x(t) = t - a, x''(t) = 0 \Rightarrow |x(a) + \|x''_{C[a,b]}\| + \|x''_{L_2[a,b]}\| = 0$. Не выполняется первое условие нормы.\\
\textbf{Ответ}\\Нет.


\section*{186}
Если $x_n \to x$, при $x\to  \infty$, то $x_n$ -- ограниченная 
последовательность.\\
\textbf{Решение}\\
Из определения сходимости $\forall \epsilon > 0\ \exists \nu(\epsilon),\ \forall n \ge \nu \Rightarrow \|x_n-x\| < \epsilon $\\
Так как $\nu(n)$ конечно, то $\exists m = \max\{\{\| x_n - x\|\ |\ i = \overline{1,\ \nu}\}\cup \{\epsilon\}\}$\\
$\forall n \Rightarrow \ \|x_n-x\| \le m$, т.е. $x_n$ содержится в шаре c радиусом $m$.

\section*{187}
$\alpha_n x_n \to \alpha x$\\
\textbf{Решение}\\
Пусть $ \alpha_n = \alpha + \Delta \alpha_n $, $ x_n = x + \Delta x_n $;\\
 $\Delta \alpha_n, \Delta x_n$ стремятся к нулю при $ n \rightarrow \infty $. 
Тогда $\| \alpha_n x_n - \alpha x \| = \| (\alpha + \Delta \alpha_n) (x + \Delta x_n) - \alpha x \|$\\
$\| \alpha \Delta x_n + \Delta \alpha_n x + \Delta \alpha_n \Delta x_n \| \le \| \alpha \Delta x_n \| + \| \Delta \alpha_n x \| + \| \Delta \alpha_n \Delta x_n \| \rightarrow 0$\\

\section*{188}
$x_n \to x \Rightarrow \|x_n\| \to \|x\|$\\
\textbf{Решение}\\
$\|x_n\| = \|x + (x_n - x)\| \le \|x\| + \|x_n - x\| \Rightarrow \|x_n\| - \|x\| \le \|x_n - x\| $\\
Аналогично, $ \|x\| - \|x_n\| \le \|x_n - x\| $.\\
Значит, $ \left|\|x_n\| - \|x\|\right| \le \|x_n - x\| $, при $ \|x_n - x\| \rightarrow 0 \quad \left|\|x_n\| - \|x\|\right| \rightarrow 0$, что и требовалось доказать.\\


\section*{189}
$x_n \to x$ и $\|x_n - y_n\| \to 0$ $\Rightarrow$ $y_n \to y$\\
\textbf{Решение}\\
$\|x_n - y_n\| = \|(x_n - x) - (y_n - y)\| \le \|x_n - x\| + \|y_n - y\|$\\
Т.к. $\|x_n - y_n\| \to 0$ и $\|x_n - x\| \to 0$, то $\|y_n - y\| \to 0$.

\section*{190}
$x_n \to x \Rightarrow \| x_n - y\| \to \| x-y \|$\\
\textbf{Решение}\\
Пусть $x_n = x + \Delta x_n$, $\| \Delta x_n \| \to 0$\\
$\| x-y\| - \|x_n\| \le \| x_n - y\| = \|x + \Delta x_n - y\| \le \|x-y\| + \|\Delta x_n\|$\\
Т.к. $\|\Delta x_n\| \to 0$, то $\|x_n-y\| \to \|x-y\|$

\section*{191}
$x_n \to x$ и $y_n \to y$, то $\|x_n - y_n\| \to \|x-y\|$\\
\textbf{Решение}\\
Пусть $x_n = x + \Delta x_n$, $\Delta x_n \to 0$ и $y_n = y + \Delta y_n$, $\Delta y_n \to 0$\\
$\| x-y\| - \| \Delta x_n - \Delta y_n \| \le \|x_n - y_n\| = \|(x-y) - (x_n - y_n)\|\le \| x - y\| + \| \Delta x_n - \Delta y_n\|$\\
Т.к. $\|\Delta x_n\| \to 0$ и $\|\Delta y_n\| \to 0$, то $\|x_n-y_n\| \to \|x-y\|$.



\end{document}
