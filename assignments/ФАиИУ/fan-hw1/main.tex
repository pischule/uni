% !TeX spellcheck = ru_RU
\documentclass{article}

\usepackage[T2A]{fontenc}
\usepackage[utf8]{inputenc}
\usepackage[russian]{babel}
\usepackage[]{amsmath}


\title{}
\author{Пищулёнок Максим}
\date{29 марта 2020 г.}


\begin{document}

\maketitle

\textbf{1.15} 
Определить, является ли множество открытым, замкнутым в пространстве $l_p$, $p \ge 1$.

\[ A = \left\{ x \in l_1 \ : x = (x_1, x_2, \dots, x_i, \dots), |x_i| \le \frac{1}{i}, i = 1, 2, \right\} \]

\emph{Решение}

Покажем, что множество $A$ не является открытым. 
Рассмотрим точку $x_0 = \left(1, \frac{1}{2}, \frac{1}{4}, \dots \right) \in A$. $\forall r > 0 \ \exists \beta(r), \ 0 < \beta < r : \ $ $y = \left( 1 + \beta, \frac{1}{2}, \frac{1}{4}, \dots \right) \in B(x_0, r)$, но $y \notin A$.

Покажем, что множество является замкнутым. Пусть $(x^n(t)) \subset A$ сходящаяся в $l_1$ последовательность. 


$\| x^n - x \| = \sum_{i = 0}^\infty |x_i^n - x_i| \xrightarrow[n \to \infty]{} 0$ 
$\Rightarrow$ 

$|x_i^n - x_i| \xrightarrow[n \to \infty]{} 0$ 
$\Rightarrow$

$|x_i| = |\lim_{n \to \infty} x_i^n| \le \frac{1}{i}$
$\Rightarrow x \in A$

\textbf{2.1} 
Определить, является ли множество выпуклым в пространстве \newline $l_p, p \ge 1$.

\[A = \left\{ A = x \in l_1 \ : x = (x_1, \dots, x_i, \dots ), |x_i| \le \frac{1}{2^{i-1}}, \ i = 1, 2, \dots \right\} \]

\emph{Решение}

Да. Множество выпукло, если вместе с каждыми точками $x, \ y \in A$ множеству принадлежит отрезок $[x, y] = \alpha x + (1 - \alpha) y$, $\alpha \in [0, \ 1]$. Пусть $z = \alpha x + (1 - \alpha) y$. Покажем, что $z \in A$.

$z = (z_1, z_2, \dots) = \alpha (x_1, x_2, \dots) + (1 - \alpha ) (y_1, y_2, \dots)$

$z_i = \alpha x_i + (1 - \alpha) y_i \le \alpha \frac{1}{2^{i-1}} + (1 - \alpha) \frac{1}{2^{i-1}} = \frac{1}{2^{i-1}}$

Следовательно, множество $A$ выпукло.

\textbf{3.15}
Образуют ли в пространстве $C[-1, 1]$ подпространство следующие множества функций:

Многочлены степени не выше 4.


\emph{Решение}

Да. Пусть $x(t), y(t) \in L $, где $L$ -- множество многочленов степени не выше 4.
Проверим, что $L$ -- линейное многообразие.

$\alpha x(t) + \beta t(t) = \alpha  \sum_{i = 0}^4 a_i t^i +  \beta \sum_{i = 0}^4 b_i t^i =  \sum_{i = 0}^4 (a_i + b_i) t^i \in L$

Проверим, что $L$ замкнуто. Пусть $x^n(t) \subset L$ -- сходящаяся к $x(t)$ последовательность.

$\| x^n(t) - x(t) \| = \max_{-1 \le t \le 1} | \sum_{i=0}^4 (a^n_i - a_i) t^i | = \\
= [k_i^n = a_i^n - a_i] = \max_{-1 \le t \le 1} |\sum k_i^n t^i| \xrightarrow[n \to \infty]{} 0$

Возьмём произвольные, попарно не равные, точки из $[-1, 1]$, например $T = \{0, 1, -1, 1/2, 1/3\}$.
 Из сходимости $x^n(t)$ следует сходимость $\sum_{i = 0}^4 k_i^n t^i, t\in T$ к нулю.

\[
\begin{cases}
	k_0^n + 0 k_1^n + 0 k_2^n + 0 k_3^n + 0 k_4^n & \xrightarrow[n \to \infty]{} 0\\
	k_0^n + k_1^n + k_2^n + k_3^n + k_4^n & \xrightarrow[n \to \infty]{} 0\\
	k_0^n - k_1^n + k_2^n - k_3^n + k_4^n & \xrightarrow[n \to \infty]{} 0\\
	k_0^n + 1/2*k_1^n + 1/4*k_2^n + 1/8*k_3^n + 1/16*k_4^n & \xrightarrow[n \to \infty]{} 0\\
	k_0^n + 1/3*k_1^n + 1/9*k_2^n + 1/27*k_3^n + 1/81*k_4^n & \xrightarrow[n \to \infty]{} 0
\end{cases}
\]

Решив систему, получим, что $k_i^n \xrightarrow[n \to \infty]{} 0 \  \forall i = \overline{0, 4}$ $\Rightarrow a_i^n \xrightarrow[n \to \infty]{} a_i \Rightarrow  x(t) \in L$

\end{document}
