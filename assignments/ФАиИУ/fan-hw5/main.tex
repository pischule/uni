\documentclass[12pt, a4paper]{article}
\usepackage[a4paper]{geometry}
\usepackage[T2A]{fontenc}
\usepackage[utf8]{inputenc}
\usepackage[russian]{babel}
\usepackage{listingsutf8}
\usepackage{float}
\usepackage{graphicx}
\usepackage{mathtools}
\usepackage{amsmath}
\usepackage{amsfonts}
\usepackage{xcolor}
\usepackage{booktabs}


\definecolor{codegreen}{rgb}{0,0.6,0}
\definecolor{codegray}{rgb}{0.5,0.5,0.5}
\definecolor{codepurple}{rgb}{0.58,0,0.82}
\definecolor{backcolour}{rgb}{0.95,0.95,0.92}

\hfuzz=30pt 

\begin{document}
\begin{titlepage}


  \newgeometry{margin=1cm}
  
  \bigskip
  \centerline{\large \bf БЕЛОРУССКИЙ ГОСУДАРСТВЕННЫЙ УНИВЕРСИТЕТ}
  \bigskip
  \bigskip  
  \centerline{\large \bf ФАКУЛЬТЕТ ПРИКЛАДНОЙ МАТЕМАТИКИ И ИНФОРМАТИКИ}
  \vfill
  \vfill
  \vfill
  \centerline{\large \bf <<Компактные множества>>}
  \bigskip
  \bigskip
  \vfill
  \begin{centering}
    {\large
    Домашняя работа\\
    студента 2 курса 2 группы \\}
  \end{centering}
  \centerline{\large \bf Пищулёнка Максима Сергеевича}
  \vfill
  \vfill
  \hfill
  \begin{minipage}{0.25\textwidth}
    {\large{\bf Преподаватель} \\
  {\it Дайняк Виктор \\ Владимирович}}
  \end{minipage}
  \vfill
  \vfill
  \centerline{\large Минск 2020}
  
  \end{titlepage}

\begin{center}
\end{center}

\textbf{Задание 1.15.} Являются ли относительно компактными следующие множества 
функций в пространстве $C[0, 1]$?

\[M = \left\{ x(t) : |x(t)|\le B\right\}\]

\textbf{Решение.}

Используем теорему Арцела-Асколи (Множество $M\in C[a, b]$ предкомпактно
тогда и только тогда, когда оно равномерно ограниченно и равностепенно 
непрерывно.)

\begin{enumerate}
  \item Покажем, что множество равномерно ограниченно. Из условия:
  
  \[\exists B = const:\ |x(t)| \le B,\ 0 \le t \le 1 \]

  \item Однако $M$ не является равностепенно непрерывным. Построим отрицание
  определения равностепенной непрерывности и покажем что её нет.

  \[ \exists x(t) \in M,\ \exists \epsilon > 0: \forall \delta > 0\ \exists 
  t_1, t_2:\ |t_1 - t_2| < \delta \rightarrow |x(t_1) - x(t_2)| \ge \epsilon \]
  
  Пусть $x(t) = \frac{x^2}{B}$.


   \begin{multline*} 
    \exists \epsilon = \frac{1}{B}: \forall \delta\ \exists t_1 = 
   \left( \frac{2}{\delta} + \frac{\delta}{2} \right),\ t_2 = \left( \frac{2}{\delta} \right): 
   |t_1 - t_2| < \delta:
   \\ |x(t_1) - x(t_2)| = \frac{1}{B} \left( \frac{4}{\delta^2} + \frac{\delta^2}{4} + 2 - \frac{4}{\delta^2} 
   \right) > \frac{1}{B}
   \end{multline*}

   \textbf{Ответ:} $M$ -- не предкомпактное множество. 

\end{enumerate}

\textbf{Задание 2.2.} Доказать, что всякое компактное множество в $C^1[0, 1]$
компактно в $C[0, 1]$.

\textbf{Решение.}

По теореме Арцела-Асколи, если множество $M$ компактно в $C^1[0, 1]$, то отсюда
следует непрерывная ограниченность и равностепенная непрерывность M в $C^1[0, 1]$.
А отсюда следует непрерывная ограниченность и равностепенная непрерывность $M$ в
$C[0, 1]$. Следовательно, $M$ является компактным в $C[0, 1]$. 

\end{document}
