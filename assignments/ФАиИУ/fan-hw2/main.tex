% !TeX spellcheck = ru_RU
\documentclass[a4paper]{article}
\usepackage[T2A]{fontenc}
\usepackage[utf8]{inputenc}
\usepackage[russian]{babel}
\usepackage[left=30mm,right=15mm, top=25mm, bottom=25mm]{geometry}
\usepackage{amsmath}
\usepackage{amssymb}



\begin{document}
\begin{titlepage}


  \newgeometry{margin=1cm}
  
  \centerline{\large \bf МИНИСТЕРСТВО ОБРАЗОВАНИЯ РЕСПУБЛИКИ БЕЛАРУСЬ}
  \bigskip
  \bigskip
  \centerline{\large \bf БЕЛОРУССКИЙ ГОСУДАРСТВЕННЫЙ УНИВЕРСИТЕТ}
  \bigskip
  \bigskip
  \centerline{\large \bf ФАКУЛЬТЕТ ПРИКЛАДНОЙ МАТЕМАТИКИ И ИНФОРМАТИКИ}
  \vfill
  \vfill
  \vfill
  \centerline{\Large \bf ФУНКЦИОНАЛЬНЫЙ АНАЛИЗ И ИНТЕГРАЛЬНЫЕ УРАВНЕНИЯ}
  \bigskip
  \bigskip
  \vfill
  \begin{centering}
    {\large
    Домашняя работа\\
    студента 2 курса 2 группы \\}
  \end{centering}
  \centerline{\large \bf Пищулёнка Максима Сергеевича}
  \vfill
  \vfill
  \hfill
  \begin{minipage}{0.25\textwidth}
    {\large{\bf Преподаватель} \\
  {\it Дайняк Виктор \\ Владимирович}}
  \end{minipage}
  \vfill
  \vfill
  \centerline{\Large \bf Минск, 2020}
  
  \end{titlepage}

\begin{center}
\end{center}

\section*{1.15}

\[F(x) = \lambda \int_0^1 t^2 s^2 x(s) ds + t^3 \]

Рассмотрим пространство $C[0, 1]$

$\| F(x) - F(y) \| = \| \lambda \int_0^1 t^2 s^2 \left( x(s) - y(s) \right) ds =$

$= \max_{0 \le t \le 1} t^2 \lambda \int_0^1  s^2 \left( x(s) - y(s) \right) ds | \le$

$\le | \lambda | \max_{0 \le t \le 1} |s|^2 |x(s) - y(s)| ds \le$

$\le | \lambda | \| x-y \| [\frac{s^3}{3}]_0^1 = \frac{|\lambda|}{3} \| x-y \| $

$\frac{|\lambda|}{3} < 1$; $-3 < \lambda < 3$

При $\lambda = \frac{1}{2}$  $F(x)$ будет сжимающим

Найдем достаточное количество итераций

$ \|x_n - a \| \le \frac{\alpha^n}{1-\alpha} \| x_0 - x_1 \|$

$\alpha = \frac{1}{6}\ x_0 = 0\ x_1 = t^3$

$\| x_n - a\| \le \frac{1/6^n}{1 - 1/6} * 1 \le 0.001$

$n = 4$ -- достаточное количество итераций

$x_2 = \frac{1}{2} t^2 \int_0^1 s^2 s^3 ds + t^3 = 
\frac{1}{2} t^2 [\frac{s^6}{6}]_0^1 + t^3 = 
\frac{t^2}{12} + t^3$

$x_3 = \frac{1}{2} t^2 \int_0^1 s^2 \left( \frac{s^2}{12} + s^3 \right) ds + t^3 = 
\frac{1}{2} t^2 [\frac{s^5}{60} + \frac{s^6}{6}]_0^1 + t^3 = \frac{11}{120} t^2 + t^3$

$x_4 = \frac{1}{2} t^2 \int_0^1 s^2 \left( \frac{11}{120} s^2 + s^3  \right)ds + t^3 =
\frac{1}{2} t^2 [\frac{11 s^5}{600} + \frac{s^6}{6}]_0^1 + t^3 = \frac{37}{400} t^2 + t^3$

Найдем точное решение уравнения 

$x(t) = \frac{t^2}{2} \int_0^1 s^2 x(s) ds + t^3$

$x(t) = \frac{t^2}{2} C + t^3$

$C = \int_0^1 s^2 x(s) ds = \int_0^1 s^2 \left( \frac{s^2}{2} C + s^3 \right) ds = $

$= \int_0^1 \left( \frac{s^4}{2} c + s^5 \right)ds = [\frac{s^5}{10} C + \frac{s^6}{6}]_0^1 =
\frac{1}{10}C + \frac{1}{6}$

$\frac{9}{10} C + \frac{1}{6}$; $C = \frac{5}{27}$

$x(t) = \frac{t^2}{2} \frac{5}{27} + t^3 = \frac{10}{108} t^2 + t^3$

$\| x(t) - x_4(t) \| = \| \frac{37}{400} - \frac{10}{108} \| =  \max_{[0, 1]} |\frac{t}{10800}|  = \frac{1}{10800} < 0.001$


Рассмотрим пространство $L_2[0, 1]$.

Оценим ядро $k(t, s) = \lambda t^2 s^2$

$\int_0^1 \int_0^1 |k(t,s)|^2 ds dt = |\lambda|^2 \int_0^1 \int_0^1 t^4 s^4 ds dt = 
|\lambda|^2 \frac{1}{25} < + \infty$

$F(x)$ отображает $L_2[0, 1]$ в себя и является сжимающим преобразованием при $|\lambda| < 5$.

При $\lambda = \frac{1}{2}$ можно применить принцип сжимающих отображений. Оценим достаточное
число итераций.

$\frac{(\frac{1}{2*5})^2}{1 - \frac{1}{25}} \left( \int_0^1 |x_1(t) - x_0(t)|^2dt \right)^{1/2} < 0.001$

$\frac{1}{9*\sqrt{7} * 10^n-1} < 0.001$

$n = 3$ -- достаточное количество итераций

\newpage

\section*{2.4}

\[ g(x) = 2x^2 + 8x + 5,  \ \ \epsilon = 0.01\]

Приведем уравнение к виду $x = F(x)$, найдем  $x_0$ и $r$ такие, что $B[x_0, r] = [a,b]$
инвариантен относительно отображения $F$. Отображение $F$ должно быть сжимающим.

$x = -\frac{1}{8}(2x^2 + 5)$

$F(x) = -\frac{1}{8}(2x^2 + 5)$

Т. к. $F$ дифференцируема, то в качестве константы Липшица возьмём 

$\alpha = \max_{a\le x \le b} |F'(x)|$

$F'(x) = -\frac{x}{2}$

Условие $|F'(x)| < 1$ выполнено, если $|x| < 2$. Выберем точку в центре промежутка $x_0 = 0$. Радиус $r$ шара, в
котором существует неподвижная точка, выберем из условий:

\[
\begin{cases}
  \|x_0 - F(x_0)\| \le r(1-\alpha (r)) \\
  \alpha(r)  < 1
\end{cases}
\]

$\alpha(r) = \max_{-r \le x \le r} |F'(x)| = \frac{r}{2}$

$x_1 = F(x_0) = -\frac{5}{8}$

\[
\begin{cases}
\frac{5}{8} \le r(1 - \frac{r}{2})\\
\frac{r}{2} < 1
\end{cases}
\]

(допустим, что система имеет решение, хотя это не так)

Выберем одно из решений этой системы.

$r = 1$

Отрезок $[-1, 1]$ инвариантен относительно отображения $F$, и на нем отображение $F$ 
является сжимающим с $\alpha = 1/2$. Оценим количество итераций, достаточных для
достижения заданной точности.

$\| x_n - a \| \le \left( \frac{1}{2} \right)^n-1 \frac{5}{8} \le \frac{1}{100}$

$n = 2$ -- минимальное количество итераций.

$x_2 = F(x_1) = -\frac{1}{9} \left( 2 (-\frac{5}{8})^2 + 5 \right) = -\frac{185}{256}$


\section*{3.1}

\[ E = C[-1; 1], f(x)(t) = \frac{1}{3} \cos(x(t)) + e^t \]

$\|F(x) - F(y)\| = \| \frac{1}{3} \cos x(t) - \frac{1}{3} \cos y(t) \| = $

$=\frac{1}{3} \| \cos x(t) - \cos y(t) \| = \frac{1}{3} \| 2 \sin \frac{y(t) + x(t)}{2} 
\sin  \frac{y(t) - x(t)}{2} \| \le$


$\le [|\sin x | \le |x|, |\sin x| \le 1] = \frac{1}{3} \| x(t) - y(t) \|$

$\alpha = \frac{1}{3} < 1 \Rightarrow$ $F$ -- сжимающее отображение

$x_0 = 0$

$x_1 = F(x_0) = \frac{1}{3} + e^t$

$x_2 = F(x_2) = \frac{1}{3} \cos \left( \frac{1}{3} + e^t \right) + e^t$

$x_3 = \frac{1}{3} \cos \left( \frac{1}{3} \cos \left( \frac{1}{3} + e^t \right) + e^t \right) + e^t$

Оценим расстояние от $x_3$ до неподвижной точки $a$

$\| x_3 - a \| \le \frac{(1/3)^3}{1 - 1/3} \| x_0 - x_1 \| = \frac{1}{18}
 \| \frac{1}{3} + e^t \| =  \frac{1}{18} \left( \frac{1}{3} + e \right) \le 0.17$


 \section*{4.1}

 \[ X = C[0, 1],\ Y = C[0,1],\ F(x) = x^3 \]

 Покажем, что условие Липшица не выполняется.

 $\forall L > 0\ \exists x=0\ y=L+1 \in C[0, 1]$, для которых
 
 $\|F(x) - F(x)\| > L \| x - y \|$

 $\|(L+1)^3\| > L \| L+1\|$

 $L^3 + 3L^2 + 3L+1 > L^2 + L$

 $L^3 + 2L^2 + 2L + 1 > 0$ -- выполнено $\forall L > 0$

 Покажем, что отображение не является равномерно-непрерывным.

 Предположим обратное. По критерию Гейне, если $x_n \to x_0$ и $y_n \to y_0$, то $\| x_n - y_n\| \xrightarrow[n\to \infty]{}0$. Покажем, что это не так. Пусть $x_n = n+ 1/n, \ y_n = n \in C[0,1]$

 $\| x_n - y_n\| = \| 1/n \| \xrightarrow[n\to \infty]{} 0$

 $\| F(x_n) - F(y_n) \| = \| 1 + 3n + 3/n\| > 1$


\end{document}
