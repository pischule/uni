\documentclass[12pt, a4paper]{article}
\usepackage[a4paper]{geometry}
\usepackage[T2A]{fontenc}
\usepackage[utf8]{inputenc}
\usepackage[russian]{babel}
\usepackage{listingsutf8}
\usepackage{float}
\usepackage{graphicx}
\usepackage{mathtools}
\usepackage{amsmath}
\usepackage{amsfonts}
\usepackage{xcolor}
\usepackage{booktabs}


\definecolor{codegreen}{rgb}{0,0.6,0}
\definecolor{codegray}{rgb}{0.5,0.5,0.5}
\definecolor{codepurple}{rgb}{0.58,0,0.82}
\definecolor{backcolour}{rgb}{0.95,0.95,0.92}

\hfuzz=30pt 

\begin{document}
\begin{titlepage}


  \newgeometry{margin=1cm}
  
  \centerline{\large \bf МИНИСТЕРСТВО ОБРАЗОВАНИЯ РЕСПУБЛИКИ БЕЛАРУСЬ}
  \bigskip
  \bigskip
  \centerline{\large \bf БЕЛОРУССКИЙ ГОСУДАРСТВЕННЫЙ УНИВЕРСИТЕТ}
  \bigskip
  \bigskip
  \centerline{\large \bf ФАКУЛЬТЕТ ПРИКЛАДНОЙ МАТЕМАТИКИ И ИНФОРМАТИКИ}
  \vfill
  \vfill
  \vfill
  \centerline{\large \bf ФУНКЦИОНАЛЬНЫЙ АНАЛИЗ И ИНТЕГРАЛЬНЫЕ УРАВНЕНИЯ}
  \bigskip
  \bigskip
  \vfill
  \begin{centering}
    {\large
    Домашняя работа\\
    студента 2 курса 2 группы \\}
  \end{centering}
  \centerline{\large \bf Пищулёнка Максима Сергеевича}
  \vfill
  \vfill
  \hfill
  \begin{minipage}{0.25\textwidth}
    {\large{\bf Преподаватель} \\
  {\it Дайняк Виктор \\ Владимирович}}
  \end{minipage}
  \vfill
  \vfill
  \centerline{\Large \bf Минск, 2020}
  
  \end{titlepage}

\begin{center}
\end{center}

\textbf{Задание 2.1} В гильбертовом пространстве $l_2$ найти проекцию элемента
$x_0 \in l_2$ на подпространство $L \subset l_2$.

\[x_0 = \left(1, \frac{1}{3}, \frac{1}{3^2},\dots , \frac{1}{3^i}, \dots \right)\]
\[L = \left\{ \alpha x + \beta y:\ 
x = \left( 1, \frac{1}{5}, \frac{1}{5^2},\dots, \frac{1}{5^i}, \dots  \right),\ 
y = \left( 1, \frac{1}{6}, \frac{1}{6^2},\dots , \frac{1}{6^i},\dots  \right)
,\ \alpha, \beta \in \mathbb{R}  \right\}\]


\textbf{Решение}

Обозначим через $z$ проекцию вектора $x_0$ на подпространство $L$, тогда 
$z=\alpha x + \beta y$ и $x_0 \bot L$, т.е. $(x_0 - z, x) = 0$ и $(x_0 - z, y)=0$.
Из условия ортогональности для определения коэффициентов $\alpha$ и $\beta$ 
получим СЛАУ:

\[
\begin{cases}
  \alpha (x, x) + \beta (y, x) = (x_0, x);\\
  \alpha (x, y) + \beta (y, y) = (x_0, y).
\end{cases}
\]

\[ (x, x) = \sum_{k=0}^\infty x_k  \cdot x_k = \sum_{k=0}^\infty \frac{1}{25^k} = \frac{25}{24} \]
\[ (x, y) = (y, x) = \sum_{k=0}^\infty y_k  \cdot x_k = \sum_{k=0}^\infty \frac{1}{30^k} = \frac{30}{29} \]
\[ (y, y) = \sum_{k=0}^\infty y_k  \cdot y_k = \sum_{k=0}^\infty \frac{1}{36^k} = \frac{36}{35} \]
\[ (x_0, x) = \sum_{k=0}^\infty x_{0k}  \cdot x_k = \sum_{k=0}^\infty \frac{1}{15^k} = \frac{15}{14} \]
\[ (x_0, y) = \sum_{k=0}^\infty x_{0k}  \cdot y_k = \sum_{k=0}^\infty \frac{1}{18^k} = \frac{18}{17} \]

Система примет вид.

\[
\begin{cases}
  \frac{25}{24}\alpha + \frac{30}{29}\beta = \frac{15}{14}\\
  \frac{30}{29}\alpha + \frac{36}{35}\beta = \frac{18}{17}
\end{cases}  
\]

Решив систему, получим:

\[\alpha = \frac{3132}{595},\ \ \beta=-\frac{145}{34}\]

\[P_L x_0 = \frac{3132}{595} x -\frac{145}{34} y\]

\textbf{Ответ: $P_L x_0 = \frac{3132}{595} x -\frac{145}{34} y$}.

\newpage

\textbf{Задание 3.1} В гильбертовом пространстве $L_2[-\pi, \pi]$ найти
проекцию функции $x_0(t)$ на подпространство $L \in L_2[-\pi, \pi]$.

\[x_0(t) = t + 8,\ L = \left\{ x(t):\ 
\int_{-\pi}^{\pi} \sin^2 tx(t) dt = 0,\ \int_{ -\pi}^{\pi} tx(t) dt = 0 \right\}\]

\textbf{Решение}

Перепишем условия, которые задают подпространство $L$:

\[a(t) = \sin^2 t,\ b(t) = t,\ L = \{x(t): (x, a) = (x, b) = 0\}\]
\[L^\bot = \{ \alpha a(t) + \beta b(t),\ \alpha, \beta \in \mathbb{R}  \}\]

Пусть $y(t)$ -- проекция $x_0$ на $L$.

\[z(t) = (x_0(t) - y(t)) \in L^{\bot},\ y(t) = x_0(t) - z(t) =
x_0(t) - \alpha a(t) - \beta b(t)\]

\[
  \begin{cases}
    (y, a) = 0,\\
    (y, b) = 0;
  \end{cases}
  \begin{cases}
    (x_0, a) - (z,a) = 0,\\
    (x_0, b) - (z, b) = 0;
  \end{cases} 
  \begin{cases} 
    (x_0, a) = \alpha (a, a) + \beta (b, a),\\
    (x_0, b) = \alpha (a, b) + \beta (b, b);
  \end{cases} 
\]

\[(x_0, a) = \int_{ - \pi}^{\pi} (t + 8) \sin^2t  dt = 8 \pi \]
\[(x_0, b) = \int_{ -\pi}^{\pi} t(t + 8)dt = \frac{2 \pi^3}{3} \]
\[(a, a) = \int_{-\pi}^{\pi}\sin^4 t dt = \frac{3\pi}{4}\]
\[(b, b) = \int_{-\pi}^{\pi}t^2 dt = \frac{2 \pi^3}{3}\]
\[(a, b) = \int_{ - \pi}^{\pi} t \sin^2 (t) dt = 0\]

\[
  \begin{cases}
    8\pi = \alpha \frac{3\pi}{4},\\
    \frac{2\pi^3}{3} = \beta \frac{2\pi^3}{3};
  \end{cases}
  \begin{cases}
    \alpha = \frac{32\pi}{3},\\
    \beta = 1.
  \end{cases}
\]

\[ y(t) = P_L x_0(t) = 8 - \frac{32\pi }{8} \sin^2 t \]


\end{document}
