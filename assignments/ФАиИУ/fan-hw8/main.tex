\documentclass[a4paper, 12pt]{report}

\usepackage[T2A]{fontenc}
\usepackage[utf8]{inputenc}
\usepackage[russian]{babel}
\usepackage{amsmath}
\usepackage{mathrsfs}
\usepackage{geometry}

\begin{document}

\begin{titlepage}
    \newgeometry{margin=1cm}
    
    \bigskip
    \centerline{\large МИНИСТЕРСТВО ОБРАЗОВАНИЯ РЕСПУБЛИКИ БЕЛАРУСЬ}
    \bigskip
    \centerline{\large БЕЛОРУССКИЙ ГОСУДАРСТВЕННЫЙ УНИВЕРСИТЕТ}
    \bigskip
    \centerline{\large Факультет прикладной математики и информатики}
    \vfill
    \vfill
    \vfill
    \centerline{\LARGE Домашнее задание}
    \bigskip
    \bigskip
    \centerline{\large На тему:}
    \centerline{\large \bf \sc Сопряженное пространство}
    \bigskip
    \vfill
    \vfill
    \hfill
    \begin{minipage}{0.3\textwidth}
        {\large{\bf Подготовил:} \\
        {\it Пищулёнок Максим \\ Сергеевич}\\
        {студент 2 курса 2 группы}}
    \end{minipage}
    \vfill
    \hfill
    \begin{minipage}{0.3\textwidth}
      {\large{\bf Проверил:} \\
    {\it Дайняк Виктор \\ Владимирович}}
    \end{minipage}
    \vfill
    \vfill
    \centerline{\large Минск, 2020}
\end{titlepage}
  
\newpage
\subsection*{Задание 1.15.} 

Выяснить, задает ли следующая формула линейный ограниченный функционал. При 
положительном ответе вычислить норму $f$ для $x(t) \in L_p [a, b], p \ge 1$.

\[f(x) = \int_{0}^{\frac{1}{2}} tx (t^2) dt +
 \int_{\frac{1}{2}}^{1} t^2 x(t) dt,\ x(t) \in L_{\frac{5}{2}} [ - 1, 1]\]

 \subsubsection*{Решение.}

 Функционал линеен в силу свойств интеграла Римана. 

 Преобразуем первый интеграл.

  \[\int_{0}^{\frac{1}{2}} tx(t^2)dt = 
  \left[ t = \sqrt{\tau},\ dt = \frac{d\tau}{2 \sqrt{\tau}} \right] = 
  \frac{1}{2} \int_{0}^{\frac{1}{\sqrt{2}}} x(\tau) d \tau\]

  Функционал примет следующий вид:

  \begin{multline*}
    f(x) = \frac{1}{2} \int_{0}^{\frac{1}{\sqrt{2}}} x(t) dt - 
    \int_{\frac{1}{2}}^{1} t^2 x(t) dt = 
    \int_{ - 1}^{1} \left( \frac{\chi_{[0, 1 / \sqrt{2}]}(t)}{2} - 
    t^2 \chi_{[1 / 2]}(t)\right) x(t) dt
  \end{multline*}


  \begin{equation*}
    y(t) = 
    \begin{cases}
      0, t \in [ - 1, 0)\\
      \frac{1}{2}, t \in [0, \frac{1}{2})\\
      \frac{1}{2} - t^2, t \in [\frac{1}{2}, \frac{1}{\sqrt{2}})\\
      - t^2, t \in [\frac{1}{\sqrt{2}}, 1]
    \end{cases}
  \end{equation*}

  \[\| f \| = \| y \| = \left( \int_{ 0}^{\frac{1}{2}} \Big| \frac{1}{2} \Big|^{\frac{5}{2}}dt +
  \int_{\frac{1}{2}}^{\frac{1}{\sqrt{2}}} \Big| \frac{1}{2} - t^2 \Big|^{\frac{5}{2}}dt +
  \int_{\frac{1}{\sqrt{2}}}^{1} \Big| t^2 \Big|^{\frac{5}{2}} dt \right)^{\frac{2}{5}} 
  \approx 0.5615\]

  \textbf{Ответ:} 0.5615.

\newpage
\subsection*{Задание 3.15}

Используя теорему об общем виде линейного ограниченного функционала в гильбертовом
пространстве, вычислить норму функционала в $L_2[-1, 1]$.

\[f(x) = \int_{ - 1}^{0} t^3 x(t^2) dt + 
3 \int_{\frac{1}{2}}^{1} tx (\sqrt{t}) dt\]

\subsubsection*{Решение}

\[f(x) = \int_{ - 1}^{0} t^3 x(t^2) dt + 
3 \int_{\frac{1}{2}}^{1} tx (\sqrt{t}) dt\]

\begin{multline*}
\int_{ -1}^0t^3 x(t^2)dt = - \int_0^1 t^3 x(t^2)dt = \\=
\left[ t^2 = \tau,\ t = \sqrt{\tau},\ dt = \frac{d \tau}{2 \sqrt{\tau}} \right] =
- \frac{1}{2}\int_{0}^{1}\tau x(\tau) d \tau
\end{multline*}

\begin{equation*}
\int_{\frac{1}{2}}^{1} tx(\sqrt{t})dt = 
\left[ \sqrt{t} = \tau,\  t = \tau^2,\ dt = 2\tau d \tau \right] = 
2 \int_{\frac{1}{4}}^{1} \tau^3 x(\tau) d \tau
\end{equation*}

\begin{multline*}
f(x) = - \frac{1}{2} \int_{0}^{1} t x(t) dt + 6 \int_{\frac{1}{4}}^{1} t^3 x(t) dt =
\int_{ - 1}^{1} \left( - \frac{1}{2} t \chi_{[0, 1]} (t) + 6 t^3 \chi_{[\frac{1}{4}, 1]} \right) x(t) dt
\end{multline*}

\begin{equation*}
  y(t) = 
  \begin{cases}
    0, t \in [ - 1, 0)\\
    - \frac{t}{2}, t \in [0, \frac{1}{4})\\
    - \frac{t}{2} + 6 t^3, t\in [\frac{1}{4}, 1]
  \end{cases}
\end{equation*}

\begin{equation*}
  \| f \| = \| y \| = \left( \int_{0}^{\frac{1}{4}} \left( - \frac{t}{2} \right)^2 dt +
  \int_{\frac{1}{4}}^{1} \left( - \frac{t}{2} + 6 t^3\right)^2 dt \right)^\frac{1}{2}
  \approx 2.006
\end{equation*}

\textbf{Ответ:} 2.006.
\newpage
\subsection*{Задание 4.15}

Вычислить норму функционала в гильбертовом пространстве $l_2$, используя теорему
Рисса

\[f(x) = \sum_{k=1}^{10} x_{k^2} - x_{101},\ \ x(x_1, x_2, \dots ) \in l_2\]

\subsubsection*{Решение.}

\[f(x) = \sum_{k=1}^{10} x_{k^2} - x_{101} = (x, y) = \sum_{k = 1}^{\infty} x_k y_k \]

\begin{equation*}
  y_i = 
  \begin{cases}
    1, i \in \{1, 4, 9, \dots , 81, 100\}\\
    - 1, i = 101\\
    0
  \end{cases}
\end{equation*}

\[\| f(x) \| = \| y_k \|_{l_2} = \left( \sum_{k = 1}^{\infty} y_k^2 \right) = \sqrt{11}\]

\textbf{Ответ: } $\sqrt{11}$.

\end{document}
