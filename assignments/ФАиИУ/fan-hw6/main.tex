\documentclass[12pt, a4paper]{report}

\usepackage[T2A]{fontenc}
\usepackage[utf8]{inputenc}
\usepackage[russian]{babel}
\usepackage{amsmath}
\usepackage{geometry}


\begin{document}


\begin{titlepage}
    \newgeometry{margin=1cm}
    
    \bigskip
    \centerline{\large МИНИСТЕРСТВО ОБРАЗОВАНИЯ РЕСПУБЛИКИ БЕЛАРУСЬ}
    \bigskip
    \centerline{\large БЕЛОРУССКИЙ ГОСУДАРСТВЕННЫЙ УНИВЕРСИТЕТ}
    \bigskip
    \centerline{\large Факультет прикладной математики и информатики}
    \vfill
    \vfill
    \vfill
    \centerline{\LARGE Домашнее задание}
    \bigskip
    \bigskip
    \centerline{\large На тему:}
    \centerline{\large \bf \sc Линейные ограниченные операторы.}
    \bigskip
    \vfill
    \vfill
    \hfill
    \begin{minipage}{0.3\textwidth}
        {\large{\bf Подготовил:} \\
        {\it Пищулёнок Максим \\ Сергеевич}\\
        {студент 2 курса 2 группы}}
    \end{minipage}
    \vfill
    \hfill
    \begin{minipage}{0.3\textwidth}
      {\large{\bf Проверил:} \\
    {\it Дайняк Виктор \\ Владимирович}}
    \end{minipage}
    \vfill
    \vfill
    \centerline{\large Минск, 2020}
\end{titlepage}


\subsection*{Задача 1.15:}
Доказать, что оператор умножения на непрерывную функцию, действующий в пространстве $X=C[a, b]$ является линейным ограниченным, найти его норму.

\[Ax(t) = \frac{t^3 + 8}{t+2}x(t),\ t\in [3, 5]\]

\subsection*{Решение:}
Оператор является линейным, если для любых $x(t), y(t)\in C[3, 5]$, и любых $\alpha, \beta \in K$ выполняется условие линейности.

\[A(\alpha x + \beta y) (t) = \frac{t^3 + 8}{t + 2}(\alpha x + \beta y)(t) = 
\alpha \frac{t^3 + 8}{t + 2}x(t) + \beta\frac{t^3+8}{t + 2}y(t) = 
\alpha A x(t) + \beta A y(t)\]

Следовательно, оператор $A$ является линейным.

Покажем, что $A$ является ограниченным оператором, т.е. $\exists c > 0$ такое, что

\[ \| A x \|_{C[3, 5]} \le c \|x\|_{C[3, 5]},\ \forall x(t) \in C[3, 5] \]

\[\|Ax\|_{C[3, 5]} \le 
\max_{3 \le t \le 5} \Big| \frac{t^3 + 8}{t + 2}\Big| \max_{3 ]le t ]le 5}|x(t)| = 
\Big| \frac{5^3 + 8 }{7} \Big| \| x \|_{C[3, 5]} =19 \|x\|_{C[3, 5]}|\]

Следовательно, $c = 19$.

Мы показали, что $A$ является линейным ограниченным оператором. Из определения нормы линейного оператора следует, что 

\[\| A \| \le 19\]

С другой стороны, взяв в качестве $x(t)$ функцию $x_0(t) = 1$, получим

\[ \| x_0 \|_{C[3, 5]}  = 1 \]
\[ \| A x_0 \|_{C[3, 5]} = 19\]

Поэтому

\[19 = \frac{\| Ax \|}{\| x \|} \le \| A \| \le 19 \Rightarrow \|A\| = 19\]


\subsection*{Задача 2.15:}
Доказать, что оператор замены переменной в пространстве $X = L_p[a, b]$ является линейным ограниченным и найти его норму.

\[X = L_{3/2}[-1, 1],\ Ax(t) = (t^5 - t^{10}x(\sqrt[5]{t}))\]

\subsection*{Решение: }

Оператор является линейным:

\newpage

\begin{multline*}
    A(\alpha x + \beta y) (t) = (t^5 - t^{10}) (\alpha x + \beta y) (\sqrt[5]{t}) =\\= 
    \alpha (t^5 - t^{10}) x(\sqrt[5]{t}) + \beta (t^5 - t^{10}) x(\sqrt[5]{t}) = 
    \alpha A x(t) + \beta A y(t)
\end{multline*}

Покажем ограниченность оператора:

\begin{multline*}
    \| Ax \| = \left( \int_{ - 1}^{1} |(t^5 - t^{10}) x(\sqrt[5]{t}) |^\frac{3}{2} dt \right)^\frac{2}{3} =
    \left[ t = \tau^5, dt = 5 \tau^4 d\tau \right] =\\=
    \left( \int_{ -1}^{1} \Big| (\tau^{25} - \tau^{50}) x(\tau)|^\frac{3}{2} 5 \tau^4 d\tau \right)^\frac{2}{3} \le
    5^\frac{2}{3} \left( \int_{ - 1}^{1} \sup_{ -1 \le \tau \le 1} |\tau^{31} - \tau^{56}|^\frac{3}{2} |x(\tau)|^\frac{3}{2} d\tau  \right)^\frac{2}{3} =\\=
    5^\frac{2}{3} \left( 2^\frac{3}{2} |x(\tau)|^\frac{3}{2}d\tau  \right)^{\frac{2}{3}} =
    2 * 5^\frac{2}{3} \| x \|
\end{multline*}

Мы показали, что $\| A x \| \le 2*5^{\frac{2}{3}} \| x \|_{L_{3/2}[-1, 1]}$, из 
этого можно сделать вывод, что $\|A\| \le 2*5^{\frac{2}{3}}$. Докажем теперь, что
$\| A \| \ge 2*5^{\frac{2}{3}}$. Для этого составим последовательность:

\begin{equation*}
    x_n(t) =
    \begin{cases}
        - 1, & - 1 \le t < - 1 + \frac{1}{n}\\
        0, & - 1 + \frac{1}{n} \le t \le 1
    \end{cases}
\end{equation*}

\[\| x_n \| = \left( \int_{ - 1}^{1} |x_n|^\frac{3}{2} dt \right)^\frac{2}{3} = 
\left( \frac{1}{n} + 1 \right)^\frac{2}{3}\]
\[ \|Ax_n\| = 5^\frac{2}{5} \left(\int_{ - 1}^{ - 1 + \frac{1}{n}} |\tau^{31} - \tau^{56}| d\tau \right)^\frac{2}{3} =
\dots \to 2 * 5^\frac{2}{3}\]
\[\frac{\| Ax_n \|}{\| x_n\|} \to 2*5^\frac{2}{3}\]

Исходя из этого, имеем:
\[\| A \| = 2*5^{\frac{2}{3}}\]

\subsection*{Задача 3.15:}
Доказать, что интегральный оператор с вырожденным ядром является линейным и ограниченным оператором, если $A: C[a, b] \to C[\alpha, \beta]$. Вычислить норму оператора.

\[A: C[-1, 3] \to C[-2, 0],\ Ax(t) = \int_{-1}^{1}(t^2 - |t| + 2)s^5 x(s)ds\]

\subsection*{Решение:}

Интегральный оператор Фредгольма в пространстве непрерывных функций линеен.
Это следует из линейности интеграла Римана.

Исследуем оператор на ограниченность.

\begin{gather*}
\| Ax(t) \| = \max_{ - 2 \le t \le 0} \int_{ - 1}^{1} (t^2 - |t| + 2)s^5 x(s) ds =\\
=\max_{ - 2\le t \le 0} \Big| \int_{ - 1}^{1} (t^2 - |t| + 2| s^5 x(s) ds \Big| \le 
\max_{ - 2 \le t \le 0} \Big| t^2 - |t| + 2 \Big| \int_{ - 1}^{1} |s|^5 |x(s)|ds \le\\
\le 4 \int_{ - 1}^{1} |s^5| \max_{ - 1 \le s \le 3} |x(s) | ds = 
8 \max_{ - 1 \le s \le 3} |x(s)| \int_{0}^{1} s^5 ds =
\frac{4}{3} \| x \|_{C[ - 1, 3]}
\end{gather*}

Т.к. $\| A x\| \le \frac{4}{3}\|x\|$, то оператор является ограниченным и 
$\| A \| \le \frac{4}{3}$ 

Покажем, что $\| A \| \ge \frac{4}{3}$. Для этого построим последовательность:

\begin{equation*}
    x_n(t) = 
\begin{cases}
    - 1, & - 1 \le t \le - \frac{1}{n}\\
    nt, & - \frac{1}{n} \le t \le \frac{1}{n} \\
    1, & \frac{1}{n} \le t \le 3
\end{cases}
\end{equation*}

\[\| x_n(t) \|_{C[ - 1, 3]} = \max_{ - 1 \le t \le 3} | x_n(t)| = 1\]

\begin{multline*}
    \| Ax \|_{C[ - 2, 0]} =
     \max_{ - 2 \le t \le 0} \Big| (t^2 - |t| + 2) 
    \left( \int_{ - 1}^{ - \frac{1}{n}} - s^5 ds + \int_{ - \frac{1}{n}}^{\frac{1}{n}} s^5 nds +
    \int_{\frac{1}{n}}^{1} s^5 ds \right) \Big| =\\=
    \frac{4}{3} - O(\frac{1}{n^6})
\end{multline*}

\[ \Rightarrow \frac{4}{3} - O(\frac{1}{n^6}) \le \| A \| \le \frac{4}{3} \Rightarrow
\| A \| = \frac{4}{3} \]

\subsection*{Задача 4.15:}

Вычислить норму оператора $A: L_p[a, b] \to L_q[\alpha, \beta]$

\[A: L_4[-1, 3] \to L_2[-2, 0],\ Ax(t)=\int_{-1}^{1}(1 - t)s^5x(s)ds\]

\subsection*{Решение:}

\begin{multline*}
\| A x \|_{L_2[-2, 0]} = \left( \int_{-2}^{0} \Big| Ax(t) \Big|^2 dt \right)^{\frac{1}{2}} =
\left( \int_{ - 2}^{0} \Big| (1 - t) \int_{ - 1}^{1} s^5 x(s) ds \Big|^2 dt \right)^\frac{1}{2} =\\
\left( \int_{ - 2}^{0} (1 - t)^2 dt \left( \int_{ - 1}^{1} s^5 x(s) ds \right)^2 \right)^\frac{1}{2} =
\sqrt{\frac{26}{3}} \Big| \int_{ - 1}^{1} s^5 x(s) ds \Big| =\\=
 \left[\text{неравенство Гёльдера для } p = \frac{4}{3}, q = 4\right] \le \\ \le
 \sqrt{\frac{26}{3}} \Big| \left( \int_{ - 1}^{1} \Big| s \Big|^\frac{20}{3} ds \right)^\frac{3}{4} 
 \left( \int_{ - 1}^{1} \Big| x(s) \Big|^4 ds \right)^\frac{1}{4} \Big| \le \\ \le
 \sqrt{\frac{26}{3}} \left( \int_{ - 1}^{1} |s|^\frac{20}{3} ds \right)^\frac{3}{4} \| x \|_{L_4[ - 1, 3]}
\end{multline*}

Мы показали, что $\| A \| \le  \sqrt{\frac{26}{3}} \left( \int_{ - 1}^{1} |s|^\frac{20}{3} ds \right)^\frac{3}{4}$

Докажем это неравенство в обратную сторону. Для этого составим функцию $x_0(t)$:

\begin{equation*}
    x_0(t) = 
    \begin{cases}
        |t|^\frac{5}{3}, & t \in [ - 1, 1]\\
        0, & t \in [1, 3]
    \end{cases}
\end{equation*}

\[ \| x_0(t) \|_{L_4[ - 1, 3]} = \left( \int_{ - 1}^{1} |t|^\frac{20}{3} dt \right)^\frac{1}{4}\]
\[ \| A x_0 \|_{L_2[ - 2, 0]} = \sqrt{\frac{26}{3}} \int_{ - 1}^{1} |s|^\frac{20}{3} ds \]

\[\frac{\| Ax_0(t)\|}{\| x_0(t) \|} = \frac{\sqrt{\frac{26}{3}} \int_{-1}^{1} |s|^\frac{20}{3} ds}{\left( \int_{ 1}^{1} |s|^\frac{20}{3} ds \right)^\frac{1}{4}} =\\
\sqrt{\frac{26}{3}} \left( \int_{ - 1}^{1} |s|^\frac{20}{3} ds \right)^\frac{3}{4} = \\
2 \sqrt{\frac{26}{3}} \left( \frac{3}{23} \right)^\frac{3}{4} \]

Получим, что $\| A \| = 2 \sqrt{\frac{26}{3}} \left( \frac{3}{23} \right)^\frac{3}{4} $


\subsection*{Задача 5.15:}

Вычислить норму оператора $A: C[a, b] \to L_p[0, 1]$

\[Ax(t) = \int_{0}^{1}(s - 1)(t+2)x(s)ds - tx(1),\ x(t) \in C[-1, 1],\ p = 3\]

\subsection*{Решение:}
Отметим, что указанный оператор как сумма двух линейных операторов является линейным.
Перейдём к доказательству ограниченности оператора. По определению ограниченности,
оценим норму

\begin{align*}
    \|Ax(t)\|_{L_2[0, 1]} = \left( \int_{0}^{1} \Big| Ax(t) \Big|^3 dt \right)^3 =\\
    \left( \int_{0}^{1} \Big| \int_{0}^{1} (s-1)(t+2)x(s)ds - tx(1) \Big|^3 dt \right)^3    
\end{align*}

Рассмотрим выражение под интегралом:

\begin{multline*}
    \Big| \int_{0}^{1} (s-1)(t+2)x(s)ds - tx(1) \Big| \le
    \Big| (t+2) \int_{0}^{1} (s-1) x(s) ds \Big| + \Big| t x(1) \| \le\\ \le
    |t+2| \max_{-1 \le s \le 1} |x(s)| \Big| \int_{0}^{1} (s-1) ds \Big| +
    |t| \max_{-1 \le s \le 1} |x(s)| = \\ =
    \| x\|_{C[-1, 1]} \left( |t+2| \frac{1}{2} + |t| \right)  
\end{multline*}

Подставим полученное выражение в оценку $\| Ax\|_{L_3[0, 1]}$

\begin{multline*}
    \| Ax(t) \| \le \left( \int_{0}^{1} \left( \| x \|_{C[-1, 1]}
    \left( \frac{|t+2|}{2} + |t|  \right) \right)^3  dt\right)^{\frac{1}{3}} =\\ =
    \| x \|_{C[-1, 1]} \left( \int_{0}^{1} \left( \frac{3t}{2} +1 \right)^3 dt \right)^{1/3} =
    \| x \|_{C[-1, 1]} \sqrt[3]{\frac{203}{32}}
\end{multline*}

Теперь докажем неравенство в обратную сторону. Для этого построим следующую последовательность:

\begin{equation*}
    x_n(t) = 
    \begin{cases}
        -1,      & -1 \le t \le 0\\
        2n-1,    & 0 \le t \le \frac{1}{n}\\
        1,       & \frac{1}{n} \le t \le 1
    \end{cases}
\end{equation*}

\[
  \|x_n(t)\|_{C[1, 1]} = 1  
\]

% \begin{multline*}
%     \| A x\| = \left( \int_{0}^{1} \Big| \| x \|_{C[ 1, 1]} \left( \frac{t + 2}{2} + |t|  \right) \Big|^3 dt \right)^{\frac{1}{3}} =\\=
%     \| x \|_{C[ - 1, 1]} \left( \int_{0}^{1} \left( \frac{3t}{2} + 1 \right)^3 dt \right)^{\frac{1}{3}}
% \end{multline*}

% \begin{gather*}
%  \Big|\int_{0}^{\frac{1}{n}} (s - 1)(t + 2)(2ns - 1)ds + \int_{\frac{1}{n}}^{1}(s - 1)(t + 2)ds - t| \Big| = \\
%  \Big| (t + 2) \left(\int_{0}^{\frac{1}{n}}(2ns^2 - s(2n + 1) + 1)ds + \int_{\frac{1}{n}}^{1} (s - 1)ds \right)- t \Big| = \\
%  \Big|(t + 2) ( - \frac{1}{2} + \frac{1}{n} - \frac{1}{3n^2} - t)\Big| =\\
%  \Big| 1 - \frac{3t}{2} + O(\frac{1}{n}) \Big|
% \end{gather*}

\[
    \| A x_n\| = \left( \int_{0}^{1} \left( \frac{3t}{2} + 1 + O\left( \frac{1}{n} \right) \right)^3 dt\right)^{\frac{1}{3}} \to 
    \left( \int_{0}^{1} \left( \frac{3t}{2} + 1 \right)^3 dt \right)^{\frac{1}{3}}
\]
\[ \frac{\|A x_0\|}{\| x_0 \|} = \left( \int_{0}^{1} \left( \frac{3t}{2} + 1 \right)^3 dt \right)^{\frac{1}{3}} =\\
\left( \int_{0}^{1} \left( \frac{27t^3}{8} + \frac{27t^2}{4} + \frac{9t}{2} + 1 \right) dt \right)^{\frac{1}{3}} =\\
\sqrt[3]{\frac{203}{32}} \]


\textbf{Задание 6.15:}

Вычислить норму оператора $A: \ell_p \to \ell_q$.

\[A:\ell_7 \to \ell_3,\ Ax = \left(\frac{x_1}{9}, \frac{x_2}{9^2}, \dots, \frac{x_k}{9^k}, \dots \right)\]

\textbf{Решение:} Оператор является ограниченным т.к.

\begin{multline*}
    \| A x \|_{\ell_3} = \left( \sum_{k = 1}^{\infty} \left( \frac{x}{9^k} \right)^3 \right)^{1/3} =
    [\text{неравенство Гёльдера } p = 7/3, q = 7/4] \le \\ \le 
    \left( \left( \sum_{k=1}^{\infty} x_k^7 \right)^{3/7} 
    \left( \sum_{k=1}^{\infty} \left( \frac{1}{9} \right)^{21k/4} \right)^{4/7} \right) =
    \| x_k \|_{\ell_7} \left( \sum_{k=1}^{\infty} \left( \frac{1}{9} \right)^{21k/4} \right)^{4/21}
\end{multline*}

Это значит, что

\[\| A \| \le \left(\sum_{k=1}^{\infty} \left( \frac{1}{9} \right)^{21k/4} \right)^{4/21} \]

Возьмём в качестве $x$ последовательность

\[x_n = \frac{1}{9^{3n/4}}\]

\begin{multline*}
    \| A x \|_{\ell_3} = \left( \sum_{k = 1}^{\infty} \left( \frac{x_k}{9^k} \right)^3 \right)^{1/3} =\\=
    \left( \left( \sum_{k=1}^{\infty} \left( \frac{1}{9} \right)^{21k/4} \right)^{3/7}
    \left( \sum_{k=1}^{\infty} \left( \frac{1}{9} \right)^{21k/4} \right)^{4/7} \right) =
    \left( \sum_{k = 1}^{\infty} \left( \frac{1}{9} \right)^{21k/4} \right)^{1/3}
\end{multline*}
\[
    \| x \|_{\ell_7} = \left( \sum_{k=1}^{\infty} \frac{1}{9^{21k/4}} \right)^{1/7}
\]

Таким образом

\[\| A \| \ge \frac{\|Ax_n\|}{\|x_n\|} = \left( \sum_{k = 1}^{\infty} \frac{1}{9^{21k/4}} \right)^{4/21} =
\left(\frac{1}{3^{21/2} - 1}\right)^{4/21} \approx 0.1111 \]

Следовательно, $\| A \| = \left(\frac{1}{3^{21/2} - 1}\right)^{4/21} \approx 0.1111$.


\end{document}
