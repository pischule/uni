\documentclass[a4paper, 12pt]{report}

\usepackage[T2A]{fontenc}
\usepackage[utf8]{inputenc}
\usepackage[russian]{babel}
\usepackage{amsmath}
\usepackage{mathrsfs}
\usepackage{geometry}

\begin{document}

\begin{titlepage}
    \newgeometry{margin=1cm}
    
    \bigskip
    \centerline{\large МИНИСТЕРСТВО ОБРАЗОВАНИЯ РЕСПУБЛИКИ БЕЛАРУСЬ}
    \bigskip
    \centerline{\large БЕЛОРУССКИЙ ГОСУДАРСТВЕННЫЙ УНИВЕРСИТЕТ}
    \bigskip
    \centerline{\large Факультет прикладной математики и информатики}
    \vfill
    \vfill
    \vfill
    \centerline{\LARGE Домашнее задание}
    \bigskip
    \bigskip
    \centerline{\large На тему:}
    \centerline{\large \bf \sc Операторы в банаховых простарнствах}
    \bigskip
    \vfill
    \vfill
    \hfill
    \begin{minipage}{0.3\textwidth}
        {\large{\bf Подготовил:} \\
        {\it Пищулёнок Максим \\ Сергеевич}\\
        {студент 2 курса 2 группы}}
    \end{minipage}
    \vfill
    \hfill
    \begin{minipage}{0.3\textwidth}
      {\large{\bf Проверил:} \\
    {\it Дайняк Виктор \\ Владимирович}}
    \end{minipage}
    \vfill
    \vfill
    \centerline{\large Минск, 2020}
\end{titlepage}
  
\subsection*{Задание 1.15}

Найти сопряженный оператор $A^*$  к оператору $A:L_2[0,1] \to L_2[0,1]$, 
действующему по следующим формулам:

\[Ax(t) = \int_{t}^{t^2}tx(s) ds - \int_0^1 \sin ts^2 x(s) ds\]

\subsection*{Решение.}

По определению сопряженного оператора имеем:

\begin{multline*}
f(Ax) = (Ax, y)_{L_2[0, 1]} = \int_{0}^{1} A x(t) y(t) dt = \\ =
 \int_{ 0}^{1} \left( \int_{t}^{t^2} t x(s) ds - \int_{0}^{1} \sin t s^2 x(s) ds \right) y(t) dt = \\ =
\int_{0}^{1} \left( \int_{t}^{t^2} t x(s) ds \right) y(t) dt - 
 \int_{0}^{1} \left( \int_{0}^{1} \sin t s^2 x(s) ds \right) y(t) dt = \\ =
 \int_{0}^{1} x(s) \left( \int_{\sqrt{s}}^{s} t y(t) dt \right) ds - 
 \int_{0}^{1} x(s) \left( \int_{0}^{1} \sin t s^2 y(t) dt \right) ds = \\ =
 \int_{0}^{1} \left( \int_{\sqrt{t}}^{t} s y(s) ds - \int_{0}^{1} \sin s t^2 y(s)ds \right) x(t) dt = 
 (x, A^* y)_{L_2[0, 1]}
\end{multline*}

Откуда

\[A^* y(t) = \int_{\sqrt{t}}^{t} s y(s) ds - \int_{0}^{1} \sin s t^2 y(s) ds\]

\textbf{Ответ:} $A^* y(t) = \int_{\sqrt{t}}^{t} s y(s) ds - \int_{0}^{1} \sin s t^2 y(s) ds$.

\newpage

\subsection*{Задание 2.15}

Найти сопряженный оператор $A^*$ к $A: \ell_2 \to \ell_2$, действующему по 
следующим формулам. Будет ли $A$ самосопряженным?

\[Ax = (x_1, 0, x_2, 0, \dots ),\ x = (x_1, x_2, \dots ) \in \ell_2\]

\subsection*{Решение.}

Применим теорему Рисса об общем виде линейного ограниченного функционала в 
гильбертовом пространстве:

\begin{multline*}
f(Ax) = (Ax, y)_{\ell_2} = \sum_{i = 1}^{\infty} x_i \overline{y_{2i - 1}} = x_1 \overline{y_1} +
0 \overline{y_2} + x_2 \overline{y_3} + \dots + x_1 \overline{y_{2i - 1}} \dots  = \\ =
A^* f(x) = (x, z)_{\ell_2}
\end{multline*}

\[z = A^* y,\ z_i = y_{2i - 1}\]

Следовательно,

\[A^* y = (y_1, y_3, y_5, \dots , y_{2i - 1}, \dots )\]

\textbf{Ответ: } $A^* y = (y_1, y_3, y_5, \dots , y_{2i - 1}, \dots )$.

\end{document}
