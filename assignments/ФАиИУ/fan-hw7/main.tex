\documentclass[a4paper, 12pt]{report}

\usepackage[T2A]{fontenc}
\usepackage[utf8]{inputenc}
\usepackage[russian]{babel}
\usepackage{amsmath}
\usepackage{mathrsfs}
\usepackage{geometry}

\begin{document}

\begin{titlepage}
    \newgeometry{margin=1cm}
    
    \bigskip
    \centerline{\large МИНИСТЕРСТВО ОБРАЗОВАНИЯ РЕСПУБЛИКИ БЕЛАРУСЬ}
    \bigskip
    \centerline{\large БЕЛОРУССКИЙ ГОСУДАРСТВЕННЫЙ УНИВЕРСИТЕТ}
    \bigskip
    \centerline{\large Факультет прикладной математики и информатики}
    \vfill
    \vfill
    \vfill
    \centerline{\LARGE Домашнее задание}
    \bigskip
    \bigskip
    \centerline{\large На тему:}
    \centerline{\large \bf \sc Обратные операторы}
    \bigskip
    \vfill
    \vfill
    \hfill
    \begin{minipage}{0.3\textwidth}
        {\large{\bf Подготовил:} \\
        {\it Пищулёнок Максим \\ Сергеевич}\\
        {студент 2 курса 2 группы}}
    \end{minipage}
    \vfill
    \hfill
    \begin{minipage}{0.3\textwidth}
      {\large{\bf Проверил:} \\
    {\it Дайняк Виктор \\ Владимирович}}
    \end{minipage}
    \vfill
    \vfill
    \centerline{\large Минск, 2020}
\end{titlepage}
  


\subsection*{Задание 1.3}
Пусть $A : L \to C[0, 1]$ Выяснить, при каких $\lambda$ к оператору $A$ существует обратный
и построить его.

\[ L = \{ x(t) \in C^1[0, 1]: x(0) = 0\}, Ax(t) = x'(t) - \lambda tx(t)\]

\subsection*{Решение.}

\[A x(t) = x'(t) - \lambda t x(t) = y(t)\]

Решим однородное уравнение.

\[x'(t) - \lambda t x(t) = 0\]
\[\frac{dx}{x} = \lambda t dt \Rightarrow\]
\[\ln x = \frac{\lambda t^2}{2} + \ln C\]
\[x = C e^{ \frac{\lambda t^2}{2}}\]

\[x' = e^{\lambda \frac{t^2}{2}} (C' + C \lambda t)\]

Подставим в неоднородное.

\[e^{ \frac{\lambda t^2}{2}} (C' + C \lambda t) - \lambda t C e^{ \frac{\lambda t^2}{2}} = y(t)\]
\[C' = y(x)e^{ - \frac{\lambda t^2}{2}}\]

Получим

\[C = (t) = \int_{0}^{t} e^{ - \frac{\lambda \tau^2}{2}} y(\tau) d \tau\]

В итоге имеем:

\[x(t) = e^{\frac{\lambda t^2}{2}} \int_{0}^{t} e^{ - \frac{\lambda \tau^2}{2}} y(\tau) d\tau =
A^{ - 1} y(t)\]

\textbf{Ответ: } $A^{ - 1} y(t) = e^{\frac{\lambda t^2}{2}} \int_{0}^{t} e^{ - \frac{\lambda \tau^2}{2}} y(\tau) d\tau$.

\subsection*{Задание 2.15}
Пусть $A: C[0, 1] \to C[0, 1]$. Используя теорему Банаха об обратном операторе,
показать, что оператор $A$ непрерывно обратим, найти $A^{-1}$.

\[Ax(t) = x(t) + \int_{0}^{1} (t^2 - 1) sx(s) ds\]

\subsection*{Решение.}

По теореме Банаха, оператор $A$ должен быть линейным ограниченным:

\begin{multline*}
A(\alpha x + \beta y)(t) = \alpha x(t) + \beta y(t) + 
\alpha \int_{0}^{1} (t^2 - 1) s x(s) ds + 
\beta \int_{0}^{1} (t^2 - 1) sx(s) ds = \\ =
\alpha A x(t) + \beta A y(t)
\end{multline*}

Линейность доказана.

\begin{multline*}
\| A x\|_{C[0, 1]} = \max_{0 \le t \le 1}
\Big| x(t) + (t^2 - 1)\int_{0}^{1} sx(s) ds \Big| \le \\ \le
\max_{0 \le t \le 1} | x(t) | + \max_{0 \le t \le 1}
| t^2 - 1| \max_{0 \le t \le 1}
| x(s) | \Big| \int_{0}^{1} s ds \Big| =
\frac{3}{2} \| x \|_{C[0, 1]}
\end{multline*}

Оператор $A$ ограничен. Значит, по теореме Банаха $A$ непрерывно обратим.

\[x(t) + (t^2 - 1) \int_{0}^{1} s x(s) ds = y(t)\]
\[x(t) = y(t) - (t^2 - 1)\int_{0}^{1} s x(s) ds = 
y(t) - (t^2 - 1) C\]

\begin{multline*}
C = \int_{0}^{1} s x(s) ds = \int_{0}^{1} s (y(s) - (s^2 - 1)c)ds =\\=
\int_{0}^{1} s y(s) ds - c \int_{0}^{1} s^3 ds + c \int_{0}^{1 }s ds =
\int_{0}^{1} s y(s) ds + \frac{c}{4}
\end{multline*}

\[\Rightarrow C = \frac{4}{3} \int_{0}^{1} s y(s) ds\]
\[x(t) = y(t) - \frac{4}{3}(t^2 - 1) \int_{0}^{1} s y(s) ds\]

\textbf{Ответ:} $A^{-1} y(t) = y(t) - \frac{4}{3}(t^2 - 1) \int_{0}^{1} s y(s) ds$.

\subsection*{Задание 3.15}

Проверить, существует ли непрерывный обратный к оператору $A: \ell_2 \to \ell_2$.
В случае положительного ответа указать его.

\[ Ax = (x_1 - x_2 + x_3, 2x_1 + 3x_2 + 4x_3, 3x_1 + x_3, x_4, \dots ) \]

\subsection*{Решение.}

Нетрудно заметить, что оператор является ограниченным. Докажем его ограниченность:

\[ Ax = (x_1 - x_2 + x_3, 2x_1 + 3x_2 + 4x_3, 3x_1 + x_3, x_4, \dots ) \]

Следовательно, существует непрерывный обратный оператор $A^{-1}$.

\begin{multline*}
    \| A x \|_{\ell_2}^2 = |x_1 - x_2 + x_3 |^2 + |2 x_1 + 3 x_2 + 4 x_3 |^2 +
    | 3 x_1 + x_3 |^2 + | x_4 |^2 + \dots \le\\
    28 |x_1|^2 + 20 |x_2|^2 + 36 |x_3|^2 + |x_4|^2 + \dots \le
    36 \sum_{k = 1}^{\infty} |x_k|^2 = 36 \| x \|_{\ell_2}^2 \\
    \Rightarrow \| Ax \|_{\ell_2} \le 6 \| x \|_{\ell_2}
\end{multline*}

\[Ax = y = (y_1, y_2, y_3, y_4, \dots )\]

\begin{equation*}
    \begin{cases}
        y_1 = x_1 - x_2 + x_3\\
        y_2 = 2 x_1 + 3 x_2 + 4 x_3\\
        y_3 = 3 x_1 + x_3\\
        y_4 = x_4\\
        \cdots 
    \end{cases}
    \Leftrightarrow
    \begin{cases}
        x_1 = - \frac{3}{16} y_1 - \frac{1}{16} y_2 + \frac{7}{16}\\
        x_2 = - \frac{5}{8} y_1 + \frac{1}{8} y_2 + \frac{1}{8} y_3\\
        x_3 = \frac{9}{16} y_1 + \frac{3}{16} y_2 - \frac{5}{16} y_3\\
        x_4 = y_4\\
        \cdots
    \end{cases}
\end{equation*}

\textbf{ Ответ: }$ A^{-1} y$ $= 
(- \frac{3}{16} y_1 - \frac{1}{16} y_2 + \frac{7}{16}$,
$- \frac{5}{8} y_1 + \frac{1}{8} y_2 + \frac{1}{8} y_3$,
$ y_1 + \frac{3}{16} y_2 - \frac{5}{16} y_3,$
$ y_4$,$ \dots)$

\subsection*{Задание 4.15}
Пусть $A: X \to Y$. Какие из операторов $A_l^{-1}, A_r^{-1}, A^{-1}$ существует? Если $A_{-1}$
существует на $\mathscr{R}(A)$, будет ли $A^{-1}$ ограничен.

\[A:m\to l_1,\ A x = \left( x_1, \frac{1}{2^2}\vec{x_2},\dots, 
\frac{1}{2^k} x_k, \dots \right)\]

\subsection*{Решение.}

Покажем, что $Ker\ A$ является нулевым вектором. Рассмотрим уравнение $A x = y$:

\[\left( x_1, \frac{1}{2^2} x_2, \dots , \frac{1}{2^k} x_k, \dots \right) = 
\left( 0, 0, \dots  \right)\]
\[( x_1, x_2, \dots\, ) = (0, 0, \dots )\]

Таким образом, уравнение $A x = y$ имеет единственное нулевое решение, а, следовательно,
существует обратный левый оператор.

Определим обратный правый оператор:

\[A_r^{ - 1} y = (y_1, 2^2 y_2, \dots , 2^k y_k, \dots )\]

Рассмотрим последовательность $y_k \in \ell_1$

\[y_k = \left( \frac{2}{3} \right)^k\]

\[\| A_r^{ - 1} y \|_m = \sup_k | A_r^{ - 1} y_k | = \sup_k \left( \frac{4}{3} \right)^k > + \infty\]

Таким образом $\| A_r^{-1}y \notin m$, но $y\in \ell_1$, т.е. множество значений
оператора A не совпадает с пространством $l_2$, поэтому не существует обратного правого
оператора к $A$. Следовательно, не существует и обратного оператора.

\textbf{Ответ:} к оператору $A$ существует левый обратный оператор, но не существует
правого обратного и обратного операторов.

\end{document}
