% !TeX spellcheck = en_US
\documentclass[12pt, a4paper]{article}
\usepackage[utf8]{inputenc}
\usepackage[russian]{babel}
\usepackage{multirow}
\usepackage{listingsutf8}
\usepackage{float}
\usepackage{graphicx}
\usepackage{mathtools}
\mathtoolsset{showonlyrefs}

\usepackage{xcolor}

\definecolor{codegreen}{rgb}{0,0.6,0}
\definecolor{codegray}{rgb}{0.5,0.5,0.5}
\definecolor{codepurple}{rgb}{0.58,0,0.82}
\definecolor{backcolour}{rgb}{0.95,0.95,0.92}

\lstdefinestyle{mystyle}{
	backgroundcolor=\color{backcolour},   
	commentstyle=\color{codegreen},
	keywordstyle=\color{magenta},
	numberstyle=\tiny\color{codegray},
	stringstyle=\color{codepurple},
	basicstyle=\ttfamily\footnotesize,
	breakatwhitespace=false,         
	breaklines=true,                 
	captionpos=b,                    
	keepspaces=true,                 
	numbers=left,                    
	numbersep=5pt,                  
	showspaces=false,                
	showstringspaces=false,
	showtabs=false,                  
	tabsize=4
}

\lstset{inputencoding=utf8/koi8-r, style=mystyle}

\begin{document}
	
	\begin{titlepage}
		\centering{
			\MakeUppercase{\textbf{БЕЛОРУССКИЙ ГОСУДАРСТВЕННЫЙ УНИВЕРСИТЕТ}} \\[0.4cm]
			
			Факультет прикладной математики и информатики \\[0.4cm]
			
			\vspace{15em}
			
			{\large\bfseries{Отчёт по лабораторной работе №1}}
			
			{\large\bfseries{<<Численное решение нелинейных уравнений>>}} \\[4cm]
			
			\noindent
			\begin{tabular}{p{0.6\textwidth}p{0.4\textwidth}}
				& Выполнил: \\
				& Пищулёнок М.\,С. \\[1cm]
				& Преподаватель: \\
				& Горбачева Ю. Н.
				
			\end{tabular}
			
			\vfill
			
			{\normalsize Минск 2020}
			
		}
	\end{titlepage}

	\tableofcontents

	 \section{Постановка задачи}
	 Решить уравнение $f(x)=0$ согласно своему варианту с точностью $\epsilon = 10^{-7}$ методом простой итерации и методом Ньютона. Корень отделяем сначала графически, затем с помощью метода половинного деления до $\epsilon = 10^{-2}$ . Если корней несколько, то необходимо найти ближайший к началу координат. Провести сравнительный анализ полученных результатов.
	 
	 \begin{equation} \label{eqn:func}
	 	f(x)=x^3 - 2x^2 -10x + 15
	 \end{equation}
	 
	 \section{Теория}
	 
	 	 \subsection{Метод простой итерации решения нелинейных уравнений}
	 
	 Рассмотрим уравнение вида
	 
	 \begin{equation} \label{eqn:1}
	 f(x)=0
	 \end{equation}
	 
	 
	 где $f$ -- некоторая заданная функция, $x$ - неизвестная величина. 
	 
	 Пусть уравнение \eqref{eqn:1} каким-либо способом приведено к виду, пригодному для итерации:
	 
	 \begin{equation} 
	 x=\phi(x)
	 \end{equation}
	 
	 Будем считать, что корень $x_\infty$ отделен и указано некоторое начальное приближение $x_0$. Тогда уточнение этого значения производят по правилу
	 
	 \begin{equation} \label{eqn:2}
	 x_{k+1} = \phi(x_k), k=0,1,\dots
	 \end{equation}
	 
	 Формула \eqref{eqn:2} задаёт вычислительный процесс метода простой итерации решения нелинейных уравнений.
	 
	 \textbf{Теорема 1.} \textit{Пусть функция $\phi(x)$ на отрезке $\Delta = {x: |x-x_0| \leq \beta}$ имеет пепрерывную производную и удовлетворяет условиям}
	 
	 \begin{equation} \label{eqn:condition1}
	 	|\phi ' (x) | \leq q < 1, x \in \Delta
	 \end{equation}
	 
	 \begin{equation} \label{eqn:condition2}
	 	|x_0 - \phi(x_0)| \leq (1-q)\delta
	 \end{equation}
	 
	 \textit{Тогда уравнение $x=\phi(x)$ имеет на отрезке $\Delta$ единственное решение, и последовательность \eqref{eqn:2}} сходится к этому решению.
	 
	 \subsection{Вычислительный процесс метода Ньютона}
	 
	 Рассмотрим уравнение вида
	 
	 \begin{equation}
	 	f(x)=0
	 \end{equation}
	 
	 где $f$ - заданная функция, $x$ - неизвестная числовая величина.
	 
	 Предположим, что каким-либо способом получено приближение $x_k$ к корню $x_\infty$ . Погрешность этого приближения обозначим через $\epsilon_k$. При известном $x_k$ поиск корня равносилен поиску погрешности $\epsilon_k$. Имеем: 
	 
	 \begin{equation} \label{eqn:3}
	 	f(x_k + \epsilon_k) = 0
	 \end{equation}
	 
	 Рассмотрим разложение левой части уравнения \eqref{eqn:3} в ряд Тейлора, взяв в разложении два первых слагаемых:
	 
	 \begin{equation}
		 f(x_k) + \epsilon_k f'(x_k) + O(\epsilon_k^2) = 0
	 \end{equation}
	 
	 Если считать величину $\epsilon_k$ небольшой по модулю и отбросить остаточный член $O(\epsilon_k^2)$, то получим приближенное равенство
	 
	 \begin{equation}
	 	f(x_k) + \epsilon_k * f'(x_k)\approx 0
	 \end{equation}
	 
	 из которого найдём некоторое приближение $\Delta x_k$ значения $\epsilon_k$:
	 
	 \begin{equation}
	 	\Delta x_k = - \frac{f(x_k)}{f'(x_k)}
	 \end{equation}
	 
	 Прибавив эту поправку $\Delta x_k$ к $x_k$ получим новое приближение:
	 
	 
	 \begin{equation} \label{eqn:newton}
	 	x_{k+1} = x_k - \frac{f(x_k)}{f'(x_k)}, k=0,1,\dots 
	 \end{equation}
	 
	 \textbf{Теорема 1.} \textit{Пусть функция $f(x)$ дважды непрерывно дифференцируема на отрезке $S_0$ и удовлетворяет условиям}
	 
	 \begin{equation} \label{eqn:newton1}
	 	2|h_0| M \leq |f'(x_0),
	 \end{equation}
	 
	 \begin{equation} \label{eqn:newton2}
	 	f(x_0) f'(x_0) \ne , \, f(x_0 + 2h_0) f'(x_0 + 2h_0) \ne 0
	 \end{equation}
	 
	 \textit{Тогда условие $f(x) = 0$ имеет на отрезке $S_0$ единственное решение, и последовательность \eqref{eqn:newton} сходится к этому решению.}
	 
	 \textbf{Замечание.} Условие \eqref{eqn:newton1} можно записать в виде
	 
	 \begin{equation} \label{eqn:newton3}
	 	\left|
	 		 \frac{f(x_0) \max_{x\in S_0} |f''(x)|}
	 		 	  {(f'(x_0))^2}
	 	\right|
	 	 < 
	 	 \frac{1}{2}
	 \end{equation}

	 \section{Листинг программы}
	 
	 \lstinputlisting[language=Python]{code/main.py}
	 
	 
	 \section{Результаты}
	 
	 Для отделения корня построим график функции. Будем находить корень $x \in [0; 2]$. В качестве начального приближения возьмём $x_0 = 2$.
	 
	 \begin{figure}[H]
	 	\includegraphics[width=12cm]{images/plot}
	 	\centering
	 	\caption{График функции $f(x)$}
	 	\label{img:1}
	 	\centering
	 \end{figure}
	 
	 
	 При помощи метода дихотомии, уточним значение корня до $\epsilon = 10^{-2}$.
	 
	 \begin{table}[H]
	 	\label{tabl}
	 	\centering
	 	\begin{tabular}{|c|c|c|c|c|c|c|c|}
	 		\hline 
	 		$k$ & $a_k$ & $b_k$ & $f(a_k)$ & $f(b_k)$ & $\frac{a_k+b_k}{2}$ & $f(\frac{a_k+b_k}{2})$ & $\epsilon_k$  \\ 
	 		\hline
			1 & 0.0000 & 2.0000 & 15.0000 & -5.0000 & 1.0000 & 4.0000 & 1.0000 \\
			\hline
			2 & 1.0000 & 2.0000 & 4.0000 & -5.0000 & 1.5000 & -1.1250 & 0.5000 \\
			\hline
			3 & 1.0000 & 1.5000 & 4.0000 & -1.1250 & 1.2500 & 1.3281 & 0.2500 \\
			\hline
			4 & 1.2500 & 1.5000 & 1.3281 & -1.1250 & 1.3750 & 0.0684 & 0.1250 \\
			\hline
			5 & 1.3750 & 1.5000 & 0.0684 & -1.1250 & 1.4375 & -0.5374 & 0.0625 \\
			\hline
			6 & 1.3750 & 1.4375 & 0.0684 & -0.5374 & 1.4062 & -0.2367 & 0.0312 \\
			\hline
			7 & 1.3750 & 1.4062 & 0.0684 & -0.2367 & 1.3906 & -0.0847 & 0.0156 \\
			\hline
			8 & 1.3750 & 1.3906 & 0.0684 & -0.0847 & 1.3828 & -0.0083 & 0.0078 \\
			\hline
	 	\end{tabular} 
		\caption{Метод половинного деления}
		\label{table:1}
	 \end{table}
 
 	Для уточнения значения корня до $\epsilon = 10^{-7}$ воспользуемся методам простых итераций и методом Ньютона.
 	
 	\textbf{Метод простых итераций.} Поскольку корень $x_\infty$ локализован на отрезке $[0,2]$, то для функции $\phi(x)$ достаточно проверить выполнение условия \eqref{eqn:condition1}.
 	
 	Приведём функцию \eqref{eqn:func} к виду, пригодному для итераций:
 	
 	
 	\begin{equation} \label{eqn:phi}
 		\phi(x) = \frac{2 x^2 - 15}{x^2 - 10}
 	\end{equation}
 	
 	Проверим выполнение условия \eqref{eqn:condition1}:
 	
 	\begin{equation} \label{eqn:dphi}
 		\phi ' (x) = - \frac{10 x}{(x^2 - 10)^2}
 	\end{equation}
 	
 	Найдём максимум \eqref{eqn:dphi} на отрезке $x \in [0;2]$
 	
 	\begin{equation}
 		\phi '' (x) \ne 0 , \, x\in [0;2] \\
 	\end{equation}
 	
 	\begin{align}
 		 \phi'(0) &= 0 & \phi'(2) &= -\frac{5}{9}
 	\end{align}
 	
 	\begin{equation}
 		q = \frac{5}{9}
 	\end{equation}
 	
 	Условие \eqref{eqn:condition1} выполняется $\Rightarrow$ итерационный процесс метода простых итераций сходится.
 	
 	\textbf{Метод Ньютона.} Для применения \eqref{eqn:newton} найдём $f'(x)$
 	
	Проверим выполнение условия сходимости метода Ньютона \eqref{eqn:newton3}:
	
	
	\begin{gather}
		f(x) = x^3 - 2x^2 - 10x + 15 \\
		f'(x) = 3x^2 - 4x - 10 \\
		f''(x) = 6x - 4
	\end{gather}
	
	\begin{gather}
		x_0 = 1.3828 \\
		\max_{x\in S_0}|f''(x) | = f''(1.3906) = 4.3436 \\
		f'(x_0) = -9.7948 \\
		f(x_0) = -0.0083 \\
	\end{gather}
 	
 	\begin{equation}
 		\left|
	 		\frac{f(x_0) \max_{x\in S_0} |f''(x)|}
	 		{(f'(x_0))^2}
 		\right|
 		 \approx 0.004 < \frac{1}{2}
 	\end{equation}
 	
 	Условие \eqref{eqn:newton3} выполнено $\Rightarrow$ метод Ньютона сходится.
 
	 \begin{table}[H]
	 	
	 	\centering
	 	\begin{tabular}{|c|c|c|c|c|}
	 		\hline
	 		\multirow{2}{*}{k} & \multicolumn{2}{c|}{Метод простой итерации} & \multicolumn{2}{c|}{Метод Ньютона} \\ \cline{2-5} 
	 		& $x_k$ &  $\epsilon_k$  & $x_k$ & $\epsilon_k$  \\ \hline
			1       & 1.3828125     & & 1.3828125     & \\ \hline
			2       & 1.3817872     & 0.0010253     & 1.3819659     & 0.0008466 \\ \hline
			3       & 1.3820038     & 0.0002166     & 1.3819660     & 0.0000002 \\ \hline
			4       & 1.3819580     & 0.0000457     & 1.3819660     & 0.0000000 \\ \hline
			5       & 1.3819677     & 0.0000097  &  & \\ \hline
			6       & 1.3819657     & 0.0000020  &  & \\ \hline
			7       & 1.3819661     & 0.0000004  &  & \\ \hline
			8       & 1.3819660     & 0.0000001  &  & \\ \hline

	 		
	 	\end{tabular}
 		\caption{Метод простой итерации и метод Ньютона}
 		\label{table:2}
	 \end{table}
	 
	 \section{Выводы}
	 
	 Метод простой итерации -- один из численных методов решения уравнений. Метод основан на принципе сжимающего отображения, который применительно к численным методам в общем виде также может называться методом простой итерации или методом последовательных приближений.
	 
	 Метод Ньютона можно трактовать как частный случай метода простой итерации с выбором функции $\phi(x)$ в виде
	 
	 \begin{eqnarray}
	 	\phi(x) = x - \frac{f(x)}{f'(x)}
	 \end{eqnarray}
\end{document}