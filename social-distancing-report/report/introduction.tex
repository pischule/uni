\chapter*{ВВЕДЕНИЕ}
\addcontentsline{toc}{chapter}{ВВЕДЕНИЕ}

В 2020 году мир охватила пандемия коронавируса, также известного как COVID-19. Это инфекционное заболевание, которое быстро распространяется при тесном контакте с зараженным человеком. Вирус передается через слюну и другие респираторные выделения, которые выделяются когда инфицированный человек кашляет, чихает или разговаривает.

Предотвращение распространения этого заболевания -- одна из самых важных задач в текущее время. С этой целью Всемирная организация здравоохранения опубликовала множество рекомендаций, среди которых социальное дистанцирование является одним из самых важных и эффективных.

Социальное дистанцирование предполагает соблюдение расстояния в 1,5 метра между людьми, благодаря которому можно предотвратить распространение большинства респираторных заболеваний.

Без каких-либо затрат эта практика позволяет сдержать темп роста распространения заболевания. Поддержание социальной дистанции может быть трудной задачей в общественных местах. Использовать людей для контроля дистанцирования опасно и непрактично. Вместо этого мы можем использовать уже существующую инфраструктуру видеонаблюдения. 

Полчая видео с камер наблюдения, мы можем непрерывно наблюдать за соблюдением социальной дистанцией между людьми. Поскольку необходимые камеры уже установлены, стоимость внедрения системы будет крайне мала, как и вероятность заражения персонала. В общих чертах, мы попытаемся построить систему оповещения о случаях нарушения социального дистанцирования.

\newpage