\documentclass{beamer}

\usepackage[T1,T2A]{fontenc}
\usepackage[utf8]{inputenc}

\usepackage[russian]{babel}
\graphicspath{ {images/} }

\usetheme{metropolis}

\hypersetup{unicode=true}

\usepackage{graphicx}

\usepackage{blindtext}

\title{Программная реализация методов машинного обучения с подкреплением на примере игры в крестики-нолики}

\subtitle{Курсовой проект}

\author{Пищулёнок Максим}
\institute[БГУ]{Белорусский государственный университет\\
Кафедра компьютерных технологий и систем}


\begin{document}

\begin{frame}
    \titlepage
\end{frame}

\begin{frame}
    \frametitle{Цель и задачи работы}

    \emph{Цель работы:} разработать программу для игры в крестики-нолики с применением машинного обучения с подкреплением.
    

    \emph{Задачи работы:}
    \begin{enumerate}
        \item Изучить методы обучения с подкреплением
        \item Научиться работать с библиотекой NumPy
        \item Реализовать метод машинного обучения
        \item Разработать графическое приложение для интерактивного взаимодействия
    \end{enumerate}

\end{frame}

\begin{frame}
    \frametitle{Использованные инструменты}

    \begin{itemize}
        \item Python – язык программирования
        \item NumPy – библиотека для работы с матрицами
        \item Tkinter – библиотека для создания графического приложения
        \item PyCharm – интегрированная среда разработки, для написания и отладки кода
    \end{itemize}
\end{frame}

\begin{frame}
    \frametitle{Обучение с подкреплением}

    \emph{Обучение с подкреплением} — способ машинного обучения, при котором система обучается, взаимодействуя с некоторой средой.

    Модель обучения с подкреплением состоит из:

    \begin{enumerate}
        \item Множества состояний окружения S
        \item Множества действий A
        \item Множества вещественных вознаграждений
    \end{enumerate}

\end{frame}

\begin{frame}
    \frametitle{Цикл обучения с подкреплением}

    \begin{figure}
        \includegraphics[width=0.6\textwidth]{rl_cycle.png}
    \end{figure}
    
\end{frame}

\begin{frame}
    \frametitle{Q-learning}

    Перед обучением, значения функции Q инициализируется произвольным образом. После этого в каждый момент времени  агент:

    \begin{itemize}
        \item выбирает действие 
        \item получает вознаграждение 
        \item переходит в новое состояние 
        \item обновляет функцию Q по формуле:
        
        \begin{equation*}
            Q^{new}(s_t, a_t) \leftarrow Q(s_t, a_t) 
            + \alpha \cdot
            \left(
            r_t + \gamma \cdot \max_q Q (s_{t+1}, a) - Q(s_t, a_t)
            \right)
        \end{equation*}

    \end{itemize}

\end{frame}

\begin{frame}
    \frametitle{Процесс обучения}

    \begin{figure}
        \includegraphics[width=0.8\textwidth]{learning.png}
    \end{figure}
\end{frame}

\begin{frame}
    \frametitle{Результат обучения}

    \begin{figure}
        \includegraphics[width=0.49\textwidth]{tie.png}
        \hfill
        \includegraphics[width=0.49\textwidth]{defeat.png}
    \end{figure}
\end{frame}

\begin{frame}
    \frametitle{Заключение}

    В ходе работы по курсовой работе, был реализован один из методом машинного обучения с подкреплением – Q-learning.
\end{frame}

\begin{frame}
    \frametitle{Список использованных источников}

    \begin{enumerate}
        \item \label{itm:sutton} Richard S. Sutton. Reinforcement Learning: An Introduction / Richard S. Sutton, Andrew G. Barto – 2nd ed. – MIT Press, Cambridge, MA, 1998. – 352 p.

    
        \item \label{itm:q-learning} Q-обучение [Электронный ресурс] / Wikipedia Project – Дата доступа: 02.08.2019. – Режим доступа: \url{https://ru.wikipedia.org/?curid=2191534&oldid=101393871}.
        
        \item \label{itm:numpy} NumPy Reference [Electronic resource] -- Mode of access: \url{https://numpy.org/doc/1.19/reference/index.html} -- Date of access: 03.12.2020 
        
        \item \label{itm:last} Крестики-нолики [Электронный ресурс] / Wikipedia Project – Дата доступа: 14.11.2020. – Режим доступа: \url{https://ru.wikipedia.org/?curid=199848&oldid=110488410}.
        
        
    \end{enumerate}
    
\end{frame}

\end{document}


