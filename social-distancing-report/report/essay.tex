\begin{center}
  \large\bfseries{РЕФЕРАТ}
\end{center}

Курсовая работа, \pageref{LastPage} стр., \totalfigures\ иллюстр., \ref{itm:last} источников.

% \begin{center}
%   \Large{\@jobtitle}
% \end{center}

\textbf{Ключевые слова:} ОБУЧЕНИЕ С ПОДКРЕПЛЕНИЕМ, \mbox{Q-ОБУЧЕНИЕ}, КРЕСТИКИ-НОЛИКИ, Q-LEARNING.
% \textbf{Ключевые слова:} обучение с подкреплением, Q-обучение, крестики-нолики.

\textbf{Объекты исследования -} алгоритмы машинного обучения с подкреплением.

\textbf{Цель исследования --} разработка программы для игры в крестики-нолики с применением машинного обучения с подкреплением.

\textbf{Методы исследования --} системный подход, изучение соответствующей литературы и электронных источников, постановка задачи и её решение.

\textbf{В результате исследования} создана программа реализующая методы машинного обучения с подкреплением.

\textbf{Области применения --} использование исходного кода программы в качестве примера реализации методов машинного обучения с подкреплением.


% \begin{center}
%   \large\bfseries{РЭФЕРАТ}
% \end{center}

% \textbf{Курсавой праект}, \pageref{LastPage} с., \totalfigures\ ілюстр., \ref{itm:last} крыніц.

% \textbf{Ключавыя словы:} НАВУЧАННЕ З ПАДМАЦАВАНЬНЕМ, \mbox{Q-НАВУЧАННЕ}, КРЫЖЫКІ-НУЛІКІ.
% % \textbf{Ключавыя словы:} навучанне з падмацаваньнем, \mbox{Q-навучанне}, крыжыкі-нулікі.

% \textbf{Аб'ект даследавання} -- алгарытмы машыннага навучання з падмацаваньнем.

% \textbf{Мэта даследавання} -- распрацоўка праграмы для гульні ў крыжыкі-нулікі з ужываннем машыннага навучання з падмацаваннем.

% \textbf{Метады даследавання} -- сістэмны падыход, вывучэнне адпаведнай літаратуры і электронных крыніц, Пастаноўка задачы і яе рашэнне.

% \textbf{У выніку даследавання} створана праграма, якая рэалізуе метады машыннага навучання з падмацаваннем.

% \textbf{Вобласць прымянення} -- выкарыстанне зыходнага кода праграмы ў якасці прыкладу рэалізацыі метадаў машыннага навучання з падмацаваннем.

% \begin{center}
%   \large\bfseries{ESSAY}
% \end{center}

% Course Project, \pageref{LastPage} p., \totalfigures\ illustr., \ref{itm:last} sources.

% \textbf{Keywords:} MACHINE LEARNING, REINFORCEMENT LEARNING, \mbox{Q-LEARNING}, \mbox{TIC-TAC-TOE}.
% % \textbf{Keywords:} machine learning, reinforcement learning, Q-learning, \mbox{Tic-Tac-Toe}.

% \textbf{The object of study} -- Reinforcement learning algorithms.

% \textbf{The purpose of research} – development of a program for playing tic-tac-toe based on reinforcement learning.

% \textbf{Research methods} -- a systematic approach, study of relevant literature and electronic sources, problem statement and it's solution.

% \textbf{As a result of research}  creation of the program that implements methods of reinforcement learning.

% \textbf{Scope} -- using the source code of the program as an example of the implementation of methods of reinforcement learning.


\newpage